%\documentstyle[11pt,epsf,french]{article}
%\topmargin -2.5cm
%\addtolength{\oddsidemargin}{-1.5cm}
%\addtolength{\evensidemargin}{-1.5cm}
%\addtolength{\textheight}{7cm}
%\addtolength{\textwidth}{4cm}



% $Author: ducasse $
% $Date: 2005/11/06 13:17:29 $
% $Revision: 1.1 $

\ifx\wholebook\relax\else
\documentclass{report}
\usepackage{times}
\usepackage{graphicx}
\usepackage{ifthen}
\usepackage{xspace}
\usepackage{alltt}
\usepackage{ifpdf}
\usepackage{ifthen}
\usepackage{amsmath}
\usepackage{a4wide}

\usepackage{amssymb}
\usepackage{amsfonts}


\graphicspath{{../figures/}{figures/}{FirstContact/}{Seaside/}{Seaside/figures/}{SmallWiki/}{AdvancedSmalltalking/}{Environment/}{BotsIncExos/}{Models/}}
\usepackage[pdftex,colorlinks=true,pdfstartview=FitV,linkcolor=blue,citecolor=blue,urlcolor=blue]{hyperref}
%%%%%%%%%%%%%%%%%%%%%%%%%%%%%%%%%%%%%%%

\usepackage{ifpdf}
\ifpdf
    \pdfoutput=1
    \DeclareGraphicsExtensions{.pdf, .jpg, .png}
\else
    \renewcommand{\includegraphics}{} % No graphics in case of latex
    \DeclareGraphicsExtensions{.eps, .jpg}
\fi


%%%%%%%%%%%%%%%%%%%%%%%%%%%%%%%%%%%%%%%
\newcommand{\mainauthor}[1]{Main Author(s): #1}
\newcommand{\ducasse}{S. Ducasse, Universit\'e de Savoie, \textsf{stephane.ducasse@univ-savoie.fr}}
\newcommand{\wuyts}{R. Wuyts, Universit\'e Libre de Bruxelles, \textsf{roel.wuyts@ulb.ac.be}}
\newcommand{\bouraqadi}{N. Bouraqadi, Universit\'e Libre de Bruxelles, \textsf{bouraqadi@ensm-douai.fr}}
\newcommand{\stinckwich}{S. Stinckwich, Universit\'e de Caen, \textsf{Serge.Stinckwich@info.unicaen.fr}}
\newcommand{\bergel}{A. Bergel, Universitaet Bern, \textsf{bergel@iam.unibe.ch}}
\newcommand{\pottier}{B. Pottier, Universit\'e de Brest, \textsf{Bernard.Pottier@univ-brest.fr}}

\newcommand{\metadata}[5]{}
%\newcommand{\metadata}[5]{\begin{quote}\emph{Currently developed on: #1, Can be used for lectures using #2, latest version #4 available on #3, contact person: #5}\end{quote}}



%%%%%%%%%%%%%%%%%%%%%%%%%%%%%%%%%%%%%%%
\newboolean{hidden}
\setboolean{hidden}{true}
\newcommand{\hidden}[1]{\ifthenelse{\boolean{hidden}}{#1}{}}


\newboolean{seevwspecific}
\setboolean{seevwspecific}{true}
\newcommand{\vwspecific}[1]{\ifthenelse{\boolean{seevwspecific}}{#1}{}}

\newboolean{seesqspecific}
\setboolean{seesqspecific}{false}
\newcommand{\sqspecific}[1]{\ifthenelse{\boolean{seesqspecific}}{#1}{}}

\newboolean{seecategoryspecific}
\setboolean{seecategoryspecific}{false}
\newcommand{\categoryspecific}[1]{\ifthenelse{\boolean{seecategoryspecific}}{#1}{}}

\newboolean{seeparcelspecific}
\setboolean{seeparcelspecific}{false}
\newcommand{\parcelspecific}[1]{\ifthenelse{\boolean{seeparcelspecific}}{#1}{}}

\newboolean{seestorespecific}
\setboolean{seestorespecific}{true}
\newcommand{\storespecific}[1]{\ifthenelse{\boolean{seestorespecific}}{#1}{}}

\newboolean{seesqueaksourcespecific}
\setboolean{seesqueaksourcespecific}{false}
\newcommand{\squeaksourcespecific}[1]{\ifthenelse{\boolean{seesqueaksourcespecific}}{#1}{}}

\newboolean{seesqueakspecific}
\setboolean{seesqueakspecific}{false}
\newcommand{\squeakspecific}[1]{\ifthenelse{\boolean{seesqueakspecific}}{#1}{}}

\newcommand{\category}[0]
{\ifthenelse{\boolean{seestorespecific}}
        {package\xspace}
        {category\xspace}}

\newcommand{\stc}[1]{{\small {\sf #1}}\xspace}
\newcommand{\ST}{{\textsc Smalltalk}\xspace}
\newcommand{\tab}{\makebox[4em]{}}
\newcommand{\ttt}[1]{{\sf #1}}
\newcommand{\chev}{\ttt{>>}}
\newcommand{\superior}{\ttt{>}}
\newcommand{\assoc}{\texttt{->}}
\newcommand{\sepa}{...}
%% to get | in code
\newcommand{\stBar}{$\mid$}

%% to get >>
%%\newcommand{\sep}{$\gg$}
\newcommand{\sep}{\texttt{>>}}
%\newcommand{\sllash}{\verb=\\=}

%%code in text
\newcommand{\ct}[1]{\textsf{#1}}

\newcommand{\stMethod}[1]{\textsf{#1}}


\newcommand{\vw}{VisualWorks\xspace}
\newcommand{\VW}{VisualWorks\xspace}
\newcommand{\sq}{Squeak\xspace}
\newcommand{\store}{Store\xspace}
\renewcommand{\chaptername}{chapter}

\newcounter{exo}
%\newcommand{\exercise}[0]
%   {\stepcounter{exo}\par\vspace{0.2cm}\noindent \textbf{Exercise: \arabic{exo}.}\xspace}


\newcommand{\block}[3]{\vspace{7pt}\noindent
	\textbf{#1~\arabic{exo}\stepcounter{exo}}#2\quad#3}
\newcommand{\exercise}[1]{\block{Exercise}{}{#1}}
\newcommand{\exercisestar}[1]{\block{Exercise}{$^\bigstar$}{#1}}
\newcommand{\question}[1]{\block{Question}{}{#1}}
\newcommand{\questionstar}[1]{\block{Question}{$^\bigstar$}{#1}}

\newcommand{\exoitem}
    {\stepcounter{exo} \item[Exercise: \arabic{exo}.]}
    
%\newcommand{\fig}[3]{
%   \begin{figure}[#3]
%   \begin{center}
%   \includegraphics[width=0.8\linewidth]{#1}
%   \caption{#2.\label{fig:#1}}
%   \end{center}
%   \end{figure}
%}


\newenvironment{scode}
{\begin{alltt}\hrule\vspace{0.1cm}\sffamily}
{\vspace{0.1cm}\hrule\end{alltt}\normalsize}


%%%%%%%%%%%%%%%%%%%%%%%%%%%%%%%%%%%%%%%%%%%%%
%%%%%% Only here for backward compatibility

\newsavebox{\fminibox}
\newlength{\fminilength}

% Fait un truc encadre
\newenvironment{fminipage}[1][\linewidth]
  {\setlength{\fminilength}{#1-2\fboxsep-2\fboxrule}
        \begin{lrbox}{\fminibox}\begin{minipage}{\fminilength}}
  { \end{minipage}\end{lrbox}\noindent\fbox{\usebox{\fminibox}}}

% Pareil mais pas encadre (a utiliser pour ne pas couper une fonction

\newenvironment{nminipage}[1][\linewidth]
  {\setlength{\fminilength}{#1}
        \begin{lrbox}{\fminibox}\begin{minipage}{\fminilength}}
  { \end{minipage}\end{lrbox}\noindent\mbox{\usebox{\fminibox}}}

% Un alltt encadre
\newenvironment{falltt}
  {\vspace*{0.3cm}\begin{fminipage}\begin{alltt}}
  {\end{alltt}\end{fminipage}\vspace*{0.3cm}}

% Un alltt pas encadre
\newenvironment{nalltt}
  {\vspace*{0.3cm}\begin{nminipage}\begin{alltt}}
  {\end{alltt}\end{nminipage}\vspace*{0.3cm}}

% Une fonction encadree
\newenvironment{ffonction}[1]
  {\begin{fonction}[#1]
        \begin{fminipage}
\begin{alltt}
\rule{\linewidth}{0.5pt}}
{\end{alltt}\end{fminipage}\end{fonction}}

\newenvironment{codeonepage}
  {\begin{nminipage}\vspace*{0.2cm}\hrule\vspace*{0.1cm}
\begin{alltt}}
  {\end{alltt} \vspace*{-0.2cm}\hrule \vspace*{0.2cm} \end{nminipage}}

\newenvironment{code}
  {\vspace*{0.1cm}\hrule\vspace*{-0.1cm}\begin{alltt}}
  {\end{alltt}\vspace*{-0.2cm}\hrule \vspace*{0.1cm}}

\newcommand{\cod}[1]{{\small\textsf{#1}\xspace}}
\newcommand{\cb}[1]{\cod{#1}}
\newcommand{\trait}[1]{\cod{#1}}
\newcommand{\cls}[1]{\cod{#1}}
\newcommand{\class}[1]{\cod{#1}}

\newcommand{\ie}{i.e.,~}

%To put a return char in a code section inside a \vwspecific, \squeakspecific, ...
\chardef\caret=`\^

\newcommand{\di}{$\gg$\xspace}
\newcommand{\cat}[1]{\texttt{#1}\xspace}

\newboolean{usesqueak}
\setboolean{usesqueak}{true}
\newcommand{\smalltalk}[2]
	{\ifthenelse{\boolean{usesqueak}}
	        {#1\xspace}
	        {#2\xspace}}
	        

%%% for chapter
\font\fonttitre=cmbx12 scaled 4200
\font\fonttitrep=cmbx12 scaled 2200
%\font\fontpartie=cmbx10 scaled 1800 % Titres des parties dans la tab des mats
%\font\fontpartiegrosse=cmbx25 scaled 2000 % Titres des parties dans le texte
%\font\fontchapitre=cmbxsl10 scaled 2500 % Titres des chapitres pas utilise
\font\fontchap=cmbx12 scaled 2200% Titres des chapitres
\font\fontchapnumero=cmbx12 scaled 4000 % Titres des chapitres
	        
%%%%%%%%%%%%%%%%%%%%%%%%%%%%%%%%%%%%%%%%%%%%%%%%% Affichage d'une tete d'un chapitre sans numero (et biblio, index)
\def\@makeschapterhead#1{%
  \vspace*{2\p@}%
  {\parindent \z@ \raggedright
    \reset@font
	\vskip 18pt
	\baselineskip=16mm
	\fontencoding{OT1}\selectfont\fontchap #1\par
	\baselineskip=4mm
    \nobreak
    \vskip 30\p@
  }}
% Macro for Chapter
\makeatletter
\def\@makechapterhead#1{%
  \vspace*{-1cm}
  \vspace*{2\p@}%
  {\parindent \z@ \raggedright \reset@font
    \ifnum \c@secnumdepth >\m@ne
       \if@mainmatter
      	\flushright{\hfill\fontencoding{T1}\selectfont\fontchap\fontchapnumero\thechapter}
	%\flushright{\LARGE\@chapapp{}\thechapter}
         \par
         \vskip 9\p@
       \fi
	 \hrule height 2pt\par
	\vskip 16pt
	\baselineskip=12mm
	\fontencoding{T1}\selectfont\fontchap #1\par
	\baselineskip=4mm
    \nobreak
    \vskip 45\p@
  }}
\makeatother


\newboolean{showcomments}
\setboolean{showcomments}{false}
\ifthenelse{\boolean{showcomments}}
  {\newcommand{\mynote}[2]{
    \fbox{\bfseries\sffamily\scriptsize#1}
    {\small$\blacktriangleright$\textsf{\emph{#2}}$\blacktriangleleft$}
    % \marginpar{\fbox{\bfseries\sffamily#1}}
   }
   \newcommand{\cvsversion}{\emph{\scriptsize$-$Id: Common.tex,v 1.2 2005/11/06 13:11:22 ducasse Exp $-$}}
  }
  {\newcommand{\mynote}[2]{}
   \newcommand{\cvsversion}{}
  }
\newcommand{\here}{\mynote{***}{CONTINUE HERE}}
\newcommand\nb[1]{\mynote{NB}{#1}}
\newcommand\fix[1]{\mynote{FIX}{#1}}
% \newcommand\todo[1]{\mynote{TO DO}{#1}}
\newcommand\ab[1]{\mynote{Alex}{#1}}
\newcommand\on[1]{\mynote{Oscar}{#1}}
\newcommand\ga[1]{\mynote{Gabi}{#1}}
\newcommand\sd[1]{\mynote{Stef}{#1}}
\newcommand\lr[1]{\mynote{Lukas}{#1}}

\newboolean{toseecomment}
\setboolean{toseecomment}{false}
%%change to false to hide comment 
\newcommand{\comment}[1]{\ifthenelse{\boolean{toseecomment}}{$\blacktriangleright$ \textit{#1}$\blacktriangleleft$}{}}

\newcommand{\commented}[1]{}


% M A C R O S
% % % % % % % % % % % % % % % % % % % % % % % % % % % % % % % % % % % % %

% text styles
%\newcommand{\code}[1]{\texttt{#1}}
%\newcommand{\class}[1]{\ct{#1}}
\newcommand{\method}[1]{\ct{\##1}}
\newcommand{\bundle}[1]{\emph{#1}}
\newcommand{\package}[1]{\emph{#1}}

%% block counter
%\newcounter{block_nr}
%\setcounter{block_nr}{1}
%% block styles
%\newcommand{\block}[3]{
%	\vspace{7pt}\noindent
%	\textbf{#1~\arabic{block_nr}\addtocounter{block_nr}{1}#2}\quad#3}
%\newcommand{\exercise}[1]{\block{Exercise}{}{#1}}
%\newcommand{\exercisestar}[1]{\block{Exercise}{$^\bigstar$}{#1}}
%\newcommand{\question}[1]{\block{Question}{}{#1}}
%\newcommand{\questionstar}[1]{\block{Question}{$^\bigstar$}{#1}}

% references
\newcommand{\secref}[1]{Section~\ref{#1}}
\newcommand{\figref}[1]{Figure~\ref{#1}}





\begin{document}
\fi

\chapter{Les objets de Smalltalk-80}

%\newtheorem{exo}{Exercice}
%\title{{\bf DEUG A MIAS/SM/TI, 2$^{\grave eme}$ ann\'{e}e}\\
%TD-TP N$^0$ 1 : {\sc Les objets de Smalltalk-80}}
%\author{}
%\date{}
%\begin{document}
%\maketitle

\mainauthor{to be fixed: \pottier }
\metadata{VisualWorks}{?Squeak/VisualWorks?}{Universit\'e de Brest ---\pottier et al. }{?1.2?}{??}
\sd{fixer la version... peut-etre avoir des tests.}



\section{Observation des objets et r\`egles de priorit\'e }

\paragraph{Vocabulaire.} Nous utilisons le terme evaluer et inspecter pour deux actions distinctes. Evaluer~: selectionner une zone et faire ``print it''. 
Inspecter~: selectionner une zone et faire ``inspect''


\subsection{Inspecter les expressions suivantes}

\begin{scode}
1
2.0
$a
'une chaine'
1@2
1.0@2.0
7/2
\end{scode}


Parmi les messages, on distingue 

\begin{description}
\item [ les messages unaires ] comme \ct{new}, \ct{sin}, \ct{sqrt}, \ct{size}, \ct{first}, \ct{last}, \ct{negated})
\item [ les messages binaires ] 
\verb?+ - * / ** // \\ < <= >  >= =  ~= == ~~  &  | @ ,?
\item [ les message \`a mot cl\'e ] comme \ct{at: put:}, \ct{x: y:}, \ct{bitOr:}, \ct{bitAnd:}
\end{description}
Dans une expression, on \'evalue en priorit\'e en respectant le parenth\'esage, les messages unaires puis binaires puis \`a mots cl\'es. Si l'expression ne comporte que des messages de m\^eme priorit\'e, l'\'evaluation se fait classiquement de la gauche vers la droite.

\subsection{Evaluer et inspecter les expressions suivantes.}

\begin{scode}
7.0/2.0
1 + 1
(1 + 1) printString
(1/2) class
\end{scode}
Expliquer pourquoi le parenth\'esage est obligatoire dans les expressions pr\'ec\'edentes.

Attention, il n'y a pas de priorit\'e entre op\'erateurs, la m\'ethode \ct{+} est juste trait\'ee comme n'importe quelle autre m\'ethode. l'\'evaluation suit l'ordre des messages. 

Evaluer~:
\begin{scode}
2 + 3*4
2 + (3*4)
2  + 1/2
2 + (1/2)
\end{scode}

\subsection{Uniformit\'e des messages.} Un m\^eme message peut \^etre adress\'e
\`a des objets de types diff\'erents. Evaluer et inspecter~:

\begin{scode}
2 sqrt
2.0 sqrt
(3/2) sqrt
(3/5) + (6/7)
\end{scode}


\subsection{Arithm\'etique exacte et conversion de type.}

 Evaluer~:
\begin{scode}
(11111111111111111111/11111111111111111112) + (1/11111111111111111112)
2r1000
16rFF
256 printStringRadix:2

(8 bitOr: 1) printStringRadix:2
(8 bitAnd: 9)
2e10 asInteger
\end{scode}
Faire le ou bit \`a bit des nombres binaires \ct{1010} et \ct{0011} et donner le r\'esultat en binaire.


\subsection{Les variables.}

Evaluer~:
\begin{scode}
\stBar x \stBar
x := 2.

\stBar x \stBar
x := 2 + 1/2.

\stBar x \stBar
x := 2 + 1/2.
x := 2.
\end{scode}
Quelle est la valeur de \ct{x} apr\`es \'evaluation de la derni\`ere portion de code?

\subsection{Les tableaux sont aussi des objets.}

 Evaluer~:
\begin{scode}
\stBar table  \stBar 
table := #(1 2.0 'trois' 444444444444444444444444444444444444).
table := #(1 3 6 9).
table first.
table last.
table reverse.

\stBar string \stBar 
string := '#(1 3 6 9)'.
string first.
string last.
string reverse.
  
#(10 20 30 40) at: 2

\stBar table \stBar 
table := #(1 3 6 9).
table at:1 put: (3/4)

\stBar string \stBar 
string := '#(1 3 6 9)'.
string at: 4 put: $r
\end{scode}



\subsection{Liste des objets et des messages.}
Donner la liste des objets et des messages d\'efinis dans les expressions ci-dessous. Expliquer le r\'esultat de l'\'evaluation.



\begin{scode}
\stBar aPoint \stBar
aPoint:= Point x:2 y:1.
aPoint x: aPoint x * 2

\stBar x \stBar
x:=1.5.
x negated rounded.
Fraction  numerator: x*2 denominator: 3 + x negated rounded.
\end{scode}



\section{Tableaux}

\begin{enumerate}

\item Multiplier  par 2 le 2 \`eme \'el\'ement d'un tableau,
\item Remplacer la valeur du 2 \`eme \'el\'ement d'un tableau par son oppos\'e.
\item  Remplacer la valeur du 3 \`eme  \'el\'ement par la valeur du 2 \`eme
\'el\'ement.
\item  Remplacer la valeur du 3 \`eme  \'el\'ement par la somme des 2 \`eme
et 3\`eme (ancienne valeur) \'el\'ements.
\item Le 2 \`eme du tableau \'etant une fraction, remplacer cette fraction par la fraction inverse dans le tableau.
\end{enumerate}

\section{Nombres}

\subsection{Maximum}

La m\'ethode \verb|max: unAutreNombre| appliqu\'e \`a un nombre
renvoie le plus grand des deux nombres.\\
Exemple~: \ct{2 max: 6} renvoie \ct{6}.

\begin{enumerate}
\item Calculer le maximum de 3 variables \ct{a b c} contenant des valeurs
quelconques.
\item Calculer le maximum de 3 variables \ct{a b c} contenant des valeurs
quelconques {\bf sans utiliser de variables interm\'ediaires}
\end{enumerate}

\subsection{Conversion Celsius-Fahrenheit}

La formule de conversion Celsius-Fahrenheit est :
C = (5/9) (F - 32).\\
Convertir une variable contenant un nombre (en degr\'es Fahrenheit),
en degr\'es Celsius.


\ifx\wholebook\relax\else\end{document}\fi








