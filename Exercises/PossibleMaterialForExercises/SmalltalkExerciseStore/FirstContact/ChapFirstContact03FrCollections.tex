% $Author: ducasse $
% $Date: 2005/05/16 13:38:07 $
% $Revision: 1.1.1.1 $

\ifx\wholebook\relax\else
\documentclass{report}
\usepackage{times}
\usepackage{graphicx}
\usepackage{ifthen}
\usepackage{xspace}
\usepackage{alltt}
\usepackage{ifpdf}
\usepackage{ifthen}
\usepackage{amsmath}
\usepackage{a4wide}

\usepackage{amssymb}
\usepackage{amsfonts}


\graphicspath{{../figures/}{figures/}{FirstContact/}{Seaside/}{Seaside/figures/}{SmallWiki/}{AdvancedSmalltalking/}{Environment/}{BotsIncExos/}{Models/}}
\usepackage[pdftex,colorlinks=true,pdfstartview=FitV,linkcolor=blue,citecolor=blue,urlcolor=blue]{hyperref}
%%%%%%%%%%%%%%%%%%%%%%%%%%%%%%%%%%%%%%%

\usepackage{ifpdf}
\ifpdf
    \pdfoutput=1
    \DeclareGraphicsExtensions{.pdf, .jpg, .png}
\else
    \renewcommand{\includegraphics}{} % No graphics in case of latex
    \DeclareGraphicsExtensions{.eps, .jpg}
\fi


%%%%%%%%%%%%%%%%%%%%%%%%%%%%%%%%%%%%%%%
\newcommand{\mainauthor}[1]{Main Author(s): #1}
\newcommand{\ducasse}{S. Ducasse, Universit\'e de Savoie, \textsf{stephane.ducasse@univ-savoie.fr}}
\newcommand{\wuyts}{R. Wuyts, Universit\'e Libre de Bruxelles, \textsf{roel.wuyts@ulb.ac.be}}
\newcommand{\bouraqadi}{N. Bouraqadi, Universit\'e Libre de Bruxelles, \textsf{bouraqadi@ensm-douai.fr}}
\newcommand{\stinckwich}{S. Stinckwich, Universit\'e de Caen, \textsf{Serge.Stinckwich@info.unicaen.fr}}
\newcommand{\bergel}{A. Bergel, Universitaet Bern, \textsf{bergel@iam.unibe.ch}}
\newcommand{\pottier}{B. Pottier, Universit\'e de Brest, \textsf{Bernard.Pottier@univ-brest.fr}}

\newcommand{\metadata}[5]{}
%\newcommand{\metadata}[5]{\begin{quote}\emph{Currently developed on: #1, Can be used for lectures using #2, latest version #4 available on #3, contact person: #5}\end{quote}}



%%%%%%%%%%%%%%%%%%%%%%%%%%%%%%%%%%%%%%%
\newboolean{hidden}
\setboolean{hidden}{true}
\newcommand{\hidden}[1]{\ifthenelse{\boolean{hidden}}{#1}{}}


\newboolean{seevwspecific}
\setboolean{seevwspecific}{true}
\newcommand{\vwspecific}[1]{\ifthenelse{\boolean{seevwspecific}}{#1}{}}

\newboolean{seesqspecific}
\setboolean{seesqspecific}{false}
\newcommand{\sqspecific}[1]{\ifthenelse{\boolean{seesqspecific}}{#1}{}}

\newboolean{seecategoryspecific}
\setboolean{seecategoryspecific}{false}
\newcommand{\categoryspecific}[1]{\ifthenelse{\boolean{seecategoryspecific}}{#1}{}}

\newboolean{seeparcelspecific}
\setboolean{seeparcelspecific}{false}
\newcommand{\parcelspecific}[1]{\ifthenelse{\boolean{seeparcelspecific}}{#1}{}}

\newboolean{seestorespecific}
\setboolean{seestorespecific}{true}
\newcommand{\storespecific}[1]{\ifthenelse{\boolean{seestorespecific}}{#1}{}}

\newboolean{seesqueaksourcespecific}
\setboolean{seesqueaksourcespecific}{false}
\newcommand{\squeaksourcespecific}[1]{\ifthenelse{\boolean{seesqueaksourcespecific}}{#1}{}}

\newboolean{seesqueakspecific}
\setboolean{seesqueakspecific}{false}
\newcommand{\squeakspecific}[1]{\ifthenelse{\boolean{seesqueakspecific}}{#1}{}}

\newcommand{\category}[0]
{\ifthenelse{\boolean{seestorespecific}}
        {package\xspace}
        {category\xspace}}

\newcommand{\stc}[1]{{\small {\sf #1}}\xspace}
\newcommand{\ST}{{\textsc Smalltalk}\xspace}
\newcommand{\tab}{\makebox[4em]{}}
\newcommand{\ttt}[1]{{\sf #1}}
\newcommand{\chev}{\ttt{>>}}
\newcommand{\superior}{\ttt{>}}
\newcommand{\assoc}{\texttt{->}}
\newcommand{\sepa}{...}
%% to get | in code
\newcommand{\stBar}{$\mid$}

%% to get >>
%%\newcommand{\sep}{$\gg$}
\newcommand{\sep}{\texttt{>>}}
%\newcommand{\sllash}{\verb=\\=}

%%code in text
\newcommand{\ct}[1]{\textsf{#1}}

\newcommand{\stMethod}[1]{\textsf{#1}}


\newcommand{\vw}{VisualWorks\xspace}
\newcommand{\VW}{VisualWorks\xspace}
\newcommand{\sq}{Squeak\xspace}
\newcommand{\store}{Store\xspace}
\renewcommand{\chaptername}{chapter}

\newcounter{exo}
%\newcommand{\exercise}[0]
%   {\stepcounter{exo}\par\vspace{0.2cm}\noindent \textbf{Exercise: \arabic{exo}.}\xspace}


\newcommand{\block}[3]{\vspace{7pt}\noindent
	\textbf{#1~\arabic{exo}\stepcounter{exo}}#2\quad#3}
\newcommand{\exercise}[1]{\block{Exercise}{}{#1}}
\newcommand{\exercisestar}[1]{\block{Exercise}{$^\bigstar$}{#1}}
\newcommand{\question}[1]{\block{Question}{}{#1}}
\newcommand{\questionstar}[1]{\block{Question}{$^\bigstar$}{#1}}

\newcommand{\exoitem}
    {\stepcounter{exo} \item[Exercise: \arabic{exo}.]}
    
%\newcommand{\fig}[3]{
%   \begin{figure}[#3]
%   \begin{center}
%   \includegraphics[width=0.8\linewidth]{#1}
%   \caption{#2.\label{fig:#1}}
%   \end{center}
%   \end{figure}
%}


\newenvironment{scode}
{\begin{alltt}\hrule\vspace{0.1cm}\sffamily}
{\vspace{0.1cm}\hrule\end{alltt}\normalsize}


%%%%%%%%%%%%%%%%%%%%%%%%%%%%%%%%%%%%%%%%%%%%%
%%%%%% Only here for backward compatibility

\newsavebox{\fminibox}
\newlength{\fminilength}

% Fait un truc encadre
\newenvironment{fminipage}[1][\linewidth]
  {\setlength{\fminilength}{#1-2\fboxsep-2\fboxrule}
        \begin{lrbox}{\fminibox}\begin{minipage}{\fminilength}}
  { \end{minipage}\end{lrbox}\noindent\fbox{\usebox{\fminibox}}}

% Pareil mais pas encadre (a utiliser pour ne pas couper une fonction

\newenvironment{nminipage}[1][\linewidth]
  {\setlength{\fminilength}{#1}
        \begin{lrbox}{\fminibox}\begin{minipage}{\fminilength}}
  { \end{minipage}\end{lrbox}\noindent\mbox{\usebox{\fminibox}}}

% Un alltt encadre
\newenvironment{falltt}
  {\vspace*{0.3cm}\begin{fminipage}\begin{alltt}}
  {\end{alltt}\end{fminipage}\vspace*{0.3cm}}

% Un alltt pas encadre
\newenvironment{nalltt}
  {\vspace*{0.3cm}\begin{nminipage}\begin{alltt}}
  {\end{alltt}\end{nminipage}\vspace*{0.3cm}}

% Une fonction encadree
\newenvironment{ffonction}[1]
  {\begin{fonction}[#1]
        \begin{fminipage}
\begin{alltt}
\rule{\linewidth}{0.5pt}}
{\end{alltt}\end{fminipage}\end{fonction}}

\newenvironment{codeonepage}
  {\begin{nminipage}\vspace*{0.2cm}\hrule\vspace*{0.1cm}
\begin{alltt}}
  {\end{alltt} \vspace*{-0.2cm}\hrule \vspace*{0.2cm} \end{nminipage}}

\newenvironment{code}
  {\vspace*{0.1cm}\hrule\vspace*{-0.1cm}\begin{alltt}}
  {\end{alltt}\vspace*{-0.2cm}\hrule \vspace*{0.1cm}}

\newcommand{\cod}[1]{{\small\textsf{#1}\xspace}}
\newcommand{\cb}[1]{\cod{#1}}
\newcommand{\trait}[1]{\cod{#1}}
\newcommand{\cls}[1]{\cod{#1}}
\newcommand{\class}[1]{\cod{#1}}

\newcommand{\ie}{i.e.,~}

%To put a return char in a code section inside a \vwspecific, \squeakspecific, ...
\chardef\caret=`\^

\newcommand{\di}{$\gg$\xspace}
\newcommand{\cat}[1]{\texttt{#1}\xspace}

\newboolean{usesqueak}
\setboolean{usesqueak}{true}
\newcommand{\smalltalk}[2]
	{\ifthenelse{\boolean{usesqueak}}
	        {#1\xspace}
	        {#2\xspace}}
	        

%%% for chapter
\font\fonttitre=cmbx12 scaled 4200
\font\fonttitrep=cmbx12 scaled 2200
%\font\fontpartie=cmbx10 scaled 1800 % Titres des parties dans la tab des mats
%\font\fontpartiegrosse=cmbx25 scaled 2000 % Titres des parties dans le texte
%\font\fontchapitre=cmbxsl10 scaled 2500 % Titres des chapitres pas utilise
\font\fontchap=cmbx12 scaled 2200% Titres des chapitres
\font\fontchapnumero=cmbx12 scaled 4000 % Titres des chapitres
	        
%%%%%%%%%%%%%%%%%%%%%%%%%%%%%%%%%%%%%%%%%%%%%%%%% Affichage d'une tete d'un chapitre sans numero (et biblio, index)
\def\@makeschapterhead#1{%
  \vspace*{2\p@}%
  {\parindent \z@ \raggedright
    \reset@font
	\vskip 18pt
	\baselineskip=16mm
	\fontencoding{OT1}\selectfont\fontchap #1\par
	\baselineskip=4mm
    \nobreak
    \vskip 30\p@
  }}
% Macro for Chapter
\makeatletter
\def\@makechapterhead#1{%
  \vspace*{-1cm}
  \vspace*{2\p@}%
  {\parindent \z@ \raggedright \reset@font
    \ifnum \c@secnumdepth >\m@ne
       \if@mainmatter
      	\flushright{\hfill\fontencoding{T1}\selectfont\fontchap\fontchapnumero\thechapter}
	%\flushright{\LARGE\@chapapp{}\thechapter}
         \par
         \vskip 9\p@
       \fi
	 \hrule height 2pt\par
	\vskip 16pt
	\baselineskip=12mm
	\fontencoding{T1}\selectfont\fontchap #1\par
	\baselineskip=4mm
    \nobreak
    \vskip 45\p@
  }}
\makeatother


\newboolean{showcomments}
\setboolean{showcomments}{false}
\ifthenelse{\boolean{showcomments}}
  {\newcommand{\mynote}[2]{
    \fbox{\bfseries\sffamily\scriptsize#1}
    {\small$\blacktriangleright$\textsf{\emph{#2}}$\blacktriangleleft$}
    % \marginpar{\fbox{\bfseries\sffamily#1}}
   }
   \newcommand{\cvsversion}{\emph{\scriptsize$-$Id: Common.tex,v 1.2 2005/11/06 13:11:22 ducasse Exp $-$}}
  }
  {\newcommand{\mynote}[2]{}
   \newcommand{\cvsversion}{}
  }
\newcommand{\here}{\mynote{***}{CONTINUE HERE}}
\newcommand\nb[1]{\mynote{NB}{#1}}
\newcommand\fix[1]{\mynote{FIX}{#1}}
% \newcommand\todo[1]{\mynote{TO DO}{#1}}
\newcommand\ab[1]{\mynote{Alex}{#1}}
\newcommand\on[1]{\mynote{Oscar}{#1}}
\newcommand\ga[1]{\mynote{Gabi}{#1}}
\newcommand\sd[1]{\mynote{Stef}{#1}}
\newcommand\lr[1]{\mynote{Lukas}{#1}}

\newboolean{toseecomment}
\setboolean{toseecomment}{false}
%%change to false to hide comment 
\newcommand{\comment}[1]{\ifthenelse{\boolean{toseecomment}}{$\blacktriangleright$ \textit{#1}$\blacktriangleleft$}{}}

\newcommand{\commented}[1]{}


% M A C R O S
% % % % % % % % % % % % % % % % % % % % % % % % % % % % % % % % % % % % %

% text styles
%\newcommand{\code}[1]{\texttt{#1}}
%\newcommand{\class}[1]{\ct{#1}}
\newcommand{\method}[1]{\ct{\##1}}
\newcommand{\bundle}[1]{\emph{#1}}
\newcommand{\package}[1]{\emph{#1}}

%% block counter
%\newcounter{block_nr}
%\setcounter{block_nr}{1}
%% block styles
%\newcommand{\block}[3]{
%	\vspace{7pt}\noindent
%	\textbf{#1~\arabic{block_nr}\addtocounter{block_nr}{1}#2}\quad#3}
%\newcommand{\exercise}[1]{\block{Exercise}{}{#1}}
%\newcommand{\exercisestar}[1]{\block{Exercise}{$^\bigstar$}{#1}}
%\newcommand{\question}[1]{\block{Question}{}{#1}}
%\newcommand{\questionstar}[1]{\block{Question}{$^\bigstar$}{#1}}

% references
\newcommand{\secref}[1]{Section~\ref{#1}}
\newcommand{\figref}[1]{Figure~\ref{#1}}





\begin{document}
\fi

\chapter{Les collections}

\mainauthor{to be fixed: \pottier, catherine dezan}
\metadata{VisualWorks}{?Squeak/VisualWorks?}{Universit\'e de Brest ---\pottier et al. }{?1.2?}{??}

%\documentstyle[11pt,epsf,french]{article}
%\topmargin -2.5cm
%\addtolength{\oddsidemargin}{-1.5cm}
%\addtolength{\evensidemargin}{-1.5cm}
%\addtolength{\textheight}{3cm}
%\addtolength{\textwidth}{2cm}

%%\newtheorem{exo}{Exercice}
%\title{\ct{DEUG A MIAS/SM/STPI, 2$^{\grave eme}$ ann\'{e}e}\\
%TD-TP N$^0$ 3: {\sc Les collections}}
%\author{}
%\date{Octobre 2003}
%%\input{psfig}
%\input{epsf}

%\begin{document}
%\maketitle

\section{Introduction~:organisation  hi\'erarchique des collections}
La figure suivante pr\'esente une partie de l'organisation hi\'erarchique des collections.
%\begin{centering}
%\begin{figure}[ht]
%\{figure=/home/penarvir/users/dezan/COURS/cours98-99/DEUGST80/TD-TP3/collections.eps,width=9cm}
%\end{figure}
%\end{centering}
%\begin{figure}[hbtp]
%\begin{center}
%\leavevmode
%\epsfxsize=9cm
%\epsfbox{/home/penarvir/users/dezan/COURS/cours98-99/DEUGST80/TD-TP3/collections.eps}
%\caption{
%L'organisation hi\'erarchique des collections.}
%\end{center}
%\end{figure}



\begin{scode}
Object ()
    Collection ()
          Bag ('contents')
          SequenceableCollection ()
                ArrayedCollection ()
                    Array ()
                    CharacterArray ()
                          String ()
                            Symbol ()
                          Text ('string' 'runs')
                    IntegerArray ()
                         ByteArray ()
                         WordArray ()
               Interval ('start' 'stop' 'step')
               OrderedCollection ('firstIndex' 'lastIndex')
                   SortedCollection ('sortBlock')
          Set ('tally')
               Dictionary ()
\end{scode}

\section{TP}
\subsection{La classe \ct{Collection}}
On trouve dans cette classe, les messages (les \'enum\'erateurs) compr\'ehensibles par toutes les sous-classes.

On retiendra les messages suivants~:
\begin{itemize}
\item les \'enum\'erateurs correspondant aux messages~: \ct{collect:}, \ct{do:}, \ct{detect:}, \ct{reject:}, \ct{select}, \ct{inject: into:}.
\item les messages de conversion~: \ct{asArray}, \ct{asBag}, \ct{asSet}, \ct{asSortedCollection}.
\item le message \ct{size} donne la taille de la collection

\end{itemize}
\subsection{La classe \ct{SequenceableCollection}}
Un ordre est d\'efini entre les \'elements de la collection.  Les objets de cette classe comprennent les messages \ct{first}, \ct{last}.

\begin{itemize}
\item La classe \ct{Array} est une collection d'\'el\'ements de taille fixe. La cr\'eation d'un object de cette classe peut se faire gr\^ace au message \ct{new:}.

Ex~: \ct{Array new: 12} cr\'ee un tableau de taille $12$ dont les \'el\'ements sont initialis\'es \`a \ct{nil}.

Les acc\`es aux \'el\'ements du tableau sont possibles avec la m\'ethode \ct{at:}. La modification d'un \'el\'ement du tableau est faite par la m\'ethode \ct{at: put:}.

\item La classe \ct{Interval} ne poss\`ede pas d'ordre explicite mais implicite. La cr\'eation d'un intervalle se fait avec les messages \ct{to:} ou \ct{to: by:} selon le pas de l'intervalle choisi.

Ex: \ct{1 to: 10} cr\'ee un intervalle avec tous les \'el\'ements de \ct{1} \`a \ct{10}.\\
\ct{1 to: 10 by: 2} cr\'ee un intervalle de \ct{1} \`a \ct{10} avec uniquement les \'el\'ements impairs.

\item La classe \ct{OrderedCollection} d\'efinit des collections de taille dynamique.  La cr\'eation est faite par le message \ct{new}, puis l'extension par l'utilisation du message \ct{add:} ou \ct{addAll:}.

Ex:~\ct{OrderedCollection new add:1; add:2} d\'efinit une collection avec les \'el\'ements \ct{1} et \ct{2}.

Le message \ct{addAll:} permet de d\'efinir une collection \`a partir de celle pass\'ee en param\`etre.

Ex: \ct{OrderedCollection new addAll: \#(\#(1 4) 5 6)} cr\'ee une instance de la classe \ct{OrderedCollection} contenant les \'el\'ements \ct{\#(1 4)}, 5 et 6.
La suppression d'un \'el\'ement dans un tableau se fait par la m\'ethode \ct{remove:}.

\item La classe \ct{SortedCollection} est une sous-classe de la classe \ct{OrderedCollection}. Les \'el\'ements de cette classe peuvent \^etre  tri\'es selon un certain crit\`ere d\'efinit dans un bloc en utilisant la m\'ethode \ct{sortBlock:}.\\
Ex~: 
\begin{scode}
\stBar aCol \stBar
aCol:= SortedCollection new.
aCol add:2; add:8; add:1.     
         ``la collection obtenue est triee dans l'ordre croissant''
aCol sortBlock: [ :elem1 :elem2 \stBar elem1 > elem2]. 
         ``on trie la collection dans l'ordre decroissant''
\end{scode}
\end{itemize}


\subsection*{Exercices}

\begin{itemize}
\exoitem Cr\'eer un tableau de taille \ct{5}, trier ses \'elements dans l'ordre d\'ecroissant et donner le tableau r\'esultat.

\exoitem soit \ct{matrice := \#(\#(1 0 0) \#(1 2 0) \#(1 2 3))}, faire la somme des \'el\'ements se trouvant sur la diagonale de cette matrice.

\exoitem D\'efinir par un bloc avec un param\`etre (repr\'esentant la taille des tableaux)les tableaux de la forme suivantes~:

\begin{scode}
\#(\#(1))                    pour taille=1
\#(\#(1) \#(1 2))                pour taille=2
\#(\#(1) \#(1 2) \#(1 2 3))          pour taille=3
\#(\#(1) \#(1 2) \#(1 2 3) \#(1 2 3 4))     pour taille=4
....
\end{scode}
\end{itemize}


\subsection{La classe \ct{Set}}

Un ensemble est une collection dont les \'el\'ements n'ont pas d'ordre, pas de
cl\'es d'acc\`es. Les doublons n'y sont donc pas autoris\'es.


Les m\'ethodes suivantes sont disponibles sur un ensemble~:
\begin{itemize}
\item
\ct{add: unElement} m\'ethode d'ajout de l'\'el\'ement \ct{unElement} (si il n'appartient pas d\'eja \`a l'ensemble)

\item \ct{remove: unElement ifAbsent: unBloc} supprime l'\'el\'ement
\ct{unElement} s'il est pr\'esent  et  \'evalue le bloc \ct{unBloc} si l'\'el\'ement est absent.

\item
\ct{includes: unElement} teste si l'\'el\'ement \ct{unElement} appartient \`a l'ensemble (renvoie \verb|true| si il s'y trouve, \verb|false| sinon).

\item
\ct{occurrencesOf: unElement} donne le nombre d'occurrences de l'\'element
\ct{unElement} (\ct{0} ou \ct{1})
\end{itemize}


\paragraph{Exemples.}
\sd{to do mettre des tests unitaires}

\begin{scode}
\stBar aSet \stBar
"Construction d'un ensemble vide"
aSet := Set new.                "Set ()."
"Ajouts"  
aSet add: 1; add: 1@3.
aSet.                     "Set (1 1@3)"
 
"Suppression"
aSet remove:3 ifAbsent:[aSet add: 4/5].
aSet.                     "Set (1@3 (4/5) 1)"

"Appartenance"  
aSet includes:1@3.              "true"
  
"Transformation d'un Array en Set"
\#(2 7 t 7 9) asSet              "Set (2 \#t 7 9)"
\#(3 6 6) asSet occurrencesOf: 6       "1"
\end{scode}

\subsection*{Exercices}

\begin{itemize}
\exoitem A partir d'un premier tableau, d\'efinir un nouveau tableau o\`u les doublons du tableau initial sont supprim\'es. On peut utiliser la m\'ethode \ct{asArray} pour transformer un receveur en tableau.

\exoitem Construire un ensemble contenant tous les nombres de \ct{1} \`a \ct{100}. 

\exoitem Transformer l'ensemble pr\'ec\'edent en rempla\c{c}ant chaque nombre par son reste de la division enti\`ere par \ct{5} (modulo \ct{5}). Quelle est la nature et la taille de cette nouvelle collection? Refaire la m\^eme op\'eration directement sur un intervalle de (\ct{1}..\ct{100}).

\end{itemize}

\subsection{La classe Dictionary}
Les dictionnaires sont des ensembles dont les \'el\'ements sont des instances de la classe {\it Association}. Une association est un couple d'objets, le premier \'el\'ement \'etant une cl\'e, le deuxi\`eme repr\'esentant une valeur.

Exemple~: \ct{2\assoc{}3} est une association dont la clef est \ct{2} et la valeur \ct{3}.\\

Quelques m\'ethodes disponibles sur les dictionnaires~:

\begin{itemize}
\item \ct{at:}$k$ renvoie la valeur r\'ef\'erenc\'ee par la cl\'e $k$
\item \ct{at:}$k$ \ct{put:}$o$ installe l'objet $o$ comme valeur \`a la cl\'e $k$
\item \ct{add:}$uneAssociation$ ajoute au dictionnaire, l'argument
$uneAssociation$ (qui est une association)
\item \ct{associations} renvoie une collection ordonn\'ee  des associations du dictionnaire
\item \ct{keys} renvoie un set contenant toutes les cl\'es
\item \ct{values} renvoie une collection ordonn\'ee comportant toutes les valeurs
\item \ct{includesKey:}$key$ teste si la cl\'e $key$ appartient au dictionnaire 
\item \ct{associationAt:}$key$ renvoie l'association dont la clef est $key$
\item \ct{keyAtValue:}$uneValeur$ renvoie une clef associ\'ee \`a la valeur $uneValeur$ pass\'ee en param\`etre
\item \ct{keysAndValuesDo:}$unBloc$ ex\'ecute le bloc $unBloc$ pour chaque association cl\'e-valeur du dictionnaire.
\end{itemize}


\subsection*{Exercices}

\begin{itemize}
\exoitem Cr\'eer un ensemble avec les \'el\'ements appartenant \`a l'intervalle (\ct{1} .. \ct{100}) et dont le reste de la division enti\`ere par \ct{5} est \'egal \`a \ct{0}. Construire l'association avec \ct{0} comme cl\'e et cet ensemble comme valeur.

\exoitem Cr\'eer un dictionnaire  o\`u les cl\'es sont  les restes de la division par $5$ des \'el\'ements appartenant \`a l'intervalle  (\ct{1} .. \ct{100}) et les valeurs sont les sous-ensembles constitu\'es d'\'elements de l'intervalle qui ont le m\^eme reste pour la division par \ct{5} (reste repr\'esent\'e par la cl\'e).

\exoitem Transformer le dictionnaire pr\'ec\'edent en un nouveau dictionnaire \`ou les sous-ensembles associ\'es aux diff\'erentes cl\'es ont \'et\'e remplac\'es par la somme de leurs \'el\'ements.

\end{itemize}

\subsection{La classe \ct{Bag}}
Une instance de la classe \ct{Bag} peut \^etre consid\'er\'e comme un \ct{Set} qui autorise les doublons. Un \ct{Bag} ne duplique pas physiquement les doublons, mais maintient le nombre d'occurrences de chaque \'el\'ement. Un Bag a un contenu qui est un dictionnaire associant  chaque \'el\'ement  \`a son nombre d'occurrences.

\ct{Bag} h\'erite des m\'ethodes classiques de la classe \ct{Collection}.

Quelques methodes sp\'ecifiques de \ct{Bag}~:
\begin{itemize}
\item
\ct{occurrencesOf:}$unElement$ donne le nombre d'occurrences de
l'\'element $unElement$ 
\item
\ct{add:}$unElement$ \ct{withOccurrences:}$nbOccurrences$ ajoute le nombre d'occurrences $nbOccurrences$ de l'objet $unElement$
\item
\ct{remove:}$unElement$ \ct{ifAbsent:}$unBloc$ supprime une occurrence
l'\'el\'ement $unElement$ s'il est pr\'esent  et 
\'evalue le bloc $unBloc$ si l'\'el\'ement est absent
\item
\ct{removeAllOccurrencesOf:}$unElement$ \ct{ifAbsent:}$unBloc$ supprime
toutes les occurrences de l'\'el\'ement $unElement$ s'il est pr\'esent  et
\'evalue le bloc $unBloc$ si l'\'el\'ement est absent
\item
\ct{valuesAndCountsDo:}$unBloc$ permet d'it\'erer le bloc $unBloc$ sur des
couples (objet nb\_occ\_de\_objet) en supposant que le bloc $unBloc$ a deux arguments (analogie avec la m\'ethode \ct{keysAndValuesDo:} des dictionnaires).
\end{itemize}

\subsection*{Exercices}
\begin{itemize}
\exoitem
R\'ecup\'erer dans un Bag les restes de la division par $2$ des nombres compris entre 1 et 100 000. Inspecter l'objet ainsi obtenu.

\exoitem Faire la somme des \'el\'ements se trouvant dans le Bag. Faire la m\^eme op\'eration sur le tableau correspondant \`a ce bag. Quel est le code qui offre le temps d'ex\'ecution le plus int\'eressant? Pour connaitre le temps d'ex\'ecution d'un code, utilisez l'expression \ct{Time millisecondsToRun:} suivi d'un param\`etre \'etant un bloc sans variable comportant le code \`a \'evaluer. Par exemple \ct{Time millisecondsToRun: [(1 to: 1000) asBag]}.

\exoitem G\'en\'eraliser  le test pr\'ec\'edent sur une collection de tableaux ayant des valeurs constantes et sur une collection de bag comportant les m\^emes \'el\'ements afin de mieux comparer les temps d'ex\'ecution pour l'op\'eration de la somme.
\end{itemize}


\section{Exercices (TD)}
\subsection{La classe \ct{SequenceableCollection}}

\begin{itemize}
\exoitem Cr\'eer en utilisant un bloc avec un param\`etre (le param\`etre caract\'erisant la taille des matrices g\'en\'er\'ees) les matrices carr\'es suivantes~:

\begin{scode}
\#(\#(1))             pour valeur=1
\#(\#(1 2) \#(1 2))          pour valeur=2
\#(\#(1 2 3) \#(1 2 3) \#(1 2 3) ) pour valeur=3
\end{scode}
 
\exoitem  D\'efinir \`a partir d'une matrice (tableau de lignes), un nouveau tableau o\`u les \'el\'ements sont les sommes des lignes du tableau  initial.

\exoitem Soit le tableau \ct{\#('toto' 'lulu' 'guillaume' 'luc' 'laurent')}, d\'efinir un nouveau tableau o\`u les noms sont tri\'es par ordre d\'ecroissant de taille.

\end{itemize}
\subsection{La classe \ct{Set}}
\begin{itemize}
\exoitem Convertir le tableau \#(1 2 3 4 2 4) en set. Quelle est la taille du r\'esultat?
\exoitem Comment afficher tous les \'el\'ements d'un set dans le Transcript?
\exoitem Cr\'eer un bloc \`a deux param\`etres --- deux sets --- qui renvoie un set correspondant \`a l'intersection des deux sets
\exoitem Cr\'eer un bloc \`a deux param\`etres --- un set et un objet quelconque ---
Si l'objet appartient \`a l'ensemble il est supprim\'e de celui-ci.
Si il n'y appartient pas, le message "l'objet monObjet n'appartient pas a
l'ensemble" est affich\'e sur le Transcript.
\end{itemize}

\subsection{La classe Bag}
\begin{itemize}
\exoitem Cr\'eer un bag et y ajouter tous les \'el\'ements du tableau \ct{\#(1 1 1 2 34)}

\exoitem Convertir le tableau \#(1 1 1 2 34) en Bag

\exoitem Ecrire un bloc avec une variable qui prend un bag et qui renvoie un bag.
Les valeurs du second bag correspondent au double des valeurs du premier bag

\exoitem Ecrire un bloc avec une variable qui prend un bag et qui renvoie un dictionnaire.
Le dictionnaire contient les associations \'el\'ement du bag \assoc nombre d'occurrences de l'\'el\'ement. 

\exoitem Modifier le bloc pr\'ec\'edent pour qu'il renvoie la somme de \underline{tous} les \'el\'ements du bag param\`etre
\end{itemize}

\subsection{La classe \ct{Dictionary}}

\begin{itemize}
\exoitem Cr\'eer un nouveau dictionnaire. Quelle est sa taille? Pourquoi?
\exoitem Y ajouter l'objet 15 \`a la cl\'e 1
\exoitem Y ajouter l'association 2 \assoc 5
\exoitem Que se passe-t-il si on ajoute l'association 2 \assoc 6?
\exoitem Inspecter les cl\'es du dictionnaire
\exoitem On d\'ecrit un parcours sous la forme d'une liste de trajets. Cr\'eer un dictionnaire dont
les cl\'es sont les points de d\'epart et les valeurs les points d'arriv\'ee \`a partir du tableau suivant:

\begin{tabular}{|l|l|}
\hline
Brest	&Quimper \\
\hline
Quimper	&Lorient \\
\hline
Nantes  &Vannes \\
\hline
Rennes	&Nantes \\
\hline
\end{tabular}
 
\exoitem Cr\'eer un block qui prend en param\`etre le dictionnaire et qui \'ecrit sur le Transcript l'ensemble des trajets. \underline{exemple:} 'De Brest je peux aller \`a Quimper'

\end{itemize}



\ifx\wholebook\relax\else\end{document}\fi
