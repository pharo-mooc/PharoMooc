% $Author: ducasse $
% $Date: 2005/09/17 16:33:53 $
% $Revision: 1.2 $
\ifx\wholebook\relax\else
\documentclass{report}
\usepackage{times}
\usepackage{graphicx}
\usepackage{ifthen}
\usepackage{xspace}
\usepackage{alltt}
\usepackage{ifpdf}
\usepackage{ifthen}
\usepackage{amsmath}
\usepackage{a4wide}

\usepackage{amssymb}
\usepackage{amsfonts}


\graphicspath{{../figures/}{figures/}{FirstContact/}{Seaside/}{Seaside/figures/}{SmallWiki/}{AdvancedSmalltalking/}{Environment/}{BotsIncExos/}{Models/}}
\usepackage[pdftex,colorlinks=true,pdfstartview=FitV,linkcolor=blue,citecolor=blue,urlcolor=blue]{hyperref}
%%%%%%%%%%%%%%%%%%%%%%%%%%%%%%%%%%%%%%%

\usepackage{ifpdf}
\ifpdf
    \pdfoutput=1
    \DeclareGraphicsExtensions{.pdf, .jpg, .png}
\else
    \renewcommand{\includegraphics}{} % No graphics in case of latex
    \DeclareGraphicsExtensions{.eps, .jpg}
\fi


%%%%%%%%%%%%%%%%%%%%%%%%%%%%%%%%%%%%%%%
\newcommand{\mainauthor}[1]{Main Author(s): #1}
\newcommand{\ducasse}{S. Ducasse, Universit\'e de Savoie, \textsf{stephane.ducasse@univ-savoie.fr}}
\newcommand{\wuyts}{R. Wuyts, Universit\'e Libre de Bruxelles, \textsf{roel.wuyts@ulb.ac.be}}
\newcommand{\bouraqadi}{N. Bouraqadi, Universit\'e Libre de Bruxelles, \textsf{bouraqadi@ensm-douai.fr}}
\newcommand{\stinckwich}{S. Stinckwich, Universit\'e de Caen, \textsf{Serge.Stinckwich@info.unicaen.fr}}
\newcommand{\bergel}{A. Bergel, Universitaet Bern, \textsf{bergel@iam.unibe.ch}}
\newcommand{\pottier}{B. Pottier, Universit\'e de Brest, \textsf{Bernard.Pottier@univ-brest.fr}}

\newcommand{\metadata}[5]{}
%\newcommand{\metadata}[5]{\begin{quote}\emph{Currently developed on: #1, Can be used for lectures using #2, latest version #4 available on #3, contact person: #5}\end{quote}}



%%%%%%%%%%%%%%%%%%%%%%%%%%%%%%%%%%%%%%%
\newboolean{hidden}
\setboolean{hidden}{true}
\newcommand{\hidden}[1]{\ifthenelse{\boolean{hidden}}{#1}{}}


\newboolean{seevwspecific}
\setboolean{seevwspecific}{true}
\newcommand{\vwspecific}[1]{\ifthenelse{\boolean{seevwspecific}}{#1}{}}

\newboolean{seesqspecific}
\setboolean{seesqspecific}{false}
\newcommand{\sqspecific}[1]{\ifthenelse{\boolean{seesqspecific}}{#1}{}}

\newboolean{seecategoryspecific}
\setboolean{seecategoryspecific}{false}
\newcommand{\categoryspecific}[1]{\ifthenelse{\boolean{seecategoryspecific}}{#1}{}}

\newboolean{seeparcelspecific}
\setboolean{seeparcelspecific}{false}
\newcommand{\parcelspecific}[1]{\ifthenelse{\boolean{seeparcelspecific}}{#1}{}}

\newboolean{seestorespecific}
\setboolean{seestorespecific}{true}
\newcommand{\storespecific}[1]{\ifthenelse{\boolean{seestorespecific}}{#1}{}}

\newboolean{seesqueaksourcespecific}
\setboolean{seesqueaksourcespecific}{false}
\newcommand{\squeaksourcespecific}[1]{\ifthenelse{\boolean{seesqueaksourcespecific}}{#1}{}}

\newboolean{seesqueakspecific}
\setboolean{seesqueakspecific}{false}
\newcommand{\squeakspecific}[1]{\ifthenelse{\boolean{seesqueakspecific}}{#1}{}}

\newcommand{\category}[0]
{\ifthenelse{\boolean{seestorespecific}}
        {package\xspace}
        {category\xspace}}

\newcommand{\stc}[1]{{\small {\sf #1}}\xspace}
\newcommand{\ST}{{\textsc Smalltalk}\xspace}
\newcommand{\tab}{\makebox[4em]{}}
\newcommand{\ttt}[1]{{\sf #1}}
\newcommand{\chev}{\ttt{>>}}
\newcommand{\superior}{\ttt{>}}
\newcommand{\assoc}{\texttt{->}}
\newcommand{\sepa}{...}
%% to get | in code
\newcommand{\stBar}{$\mid$}

%% to get >>
%%\newcommand{\sep}{$\gg$}
\newcommand{\sep}{\texttt{>>}}
%\newcommand{\sllash}{\verb=\\=}

%%code in text
\newcommand{\ct}[1]{\textsf{#1}}

\newcommand{\stMethod}[1]{\textsf{#1}}


\newcommand{\vw}{VisualWorks\xspace}
\newcommand{\VW}{VisualWorks\xspace}
\newcommand{\sq}{Squeak\xspace}
\newcommand{\store}{Store\xspace}
\renewcommand{\chaptername}{chapter}

\newcounter{exo}
%\newcommand{\exercise}[0]
%   {\stepcounter{exo}\par\vspace{0.2cm}\noindent \textbf{Exercise: \arabic{exo}.}\xspace}


\newcommand{\block}[3]{\vspace{7pt}\noindent
	\textbf{#1~\arabic{exo}\stepcounter{exo}}#2\quad#3}
\newcommand{\exercise}[1]{\block{Exercise}{}{#1}}
\newcommand{\exercisestar}[1]{\block{Exercise}{$^\bigstar$}{#1}}
\newcommand{\question}[1]{\block{Question}{}{#1}}
\newcommand{\questionstar}[1]{\block{Question}{$^\bigstar$}{#1}}

\newcommand{\exoitem}
    {\stepcounter{exo} \item[Exercise: \arabic{exo}.]}
    
%\newcommand{\fig}[3]{
%   \begin{figure}[#3]
%   \begin{center}
%   \includegraphics[width=0.8\linewidth]{#1}
%   \caption{#2.\label{fig:#1}}
%   \end{center}
%   \end{figure}
%}


\newenvironment{scode}
{\begin{alltt}\hrule\vspace{0.1cm}\sffamily}
{\vspace{0.1cm}\hrule\end{alltt}\normalsize}


%%%%%%%%%%%%%%%%%%%%%%%%%%%%%%%%%%%%%%%%%%%%%
%%%%%% Only here for backward compatibility

\newsavebox{\fminibox}
\newlength{\fminilength}

% Fait un truc encadre
\newenvironment{fminipage}[1][\linewidth]
  {\setlength{\fminilength}{#1-2\fboxsep-2\fboxrule}
        \begin{lrbox}{\fminibox}\begin{minipage}{\fminilength}}
  { \end{minipage}\end{lrbox}\noindent\fbox{\usebox{\fminibox}}}

% Pareil mais pas encadre (a utiliser pour ne pas couper une fonction

\newenvironment{nminipage}[1][\linewidth]
  {\setlength{\fminilength}{#1}
        \begin{lrbox}{\fminibox}\begin{minipage}{\fminilength}}
  { \end{minipage}\end{lrbox}\noindent\mbox{\usebox{\fminibox}}}

% Un alltt encadre
\newenvironment{falltt}
  {\vspace*{0.3cm}\begin{fminipage}\begin{alltt}}
  {\end{alltt}\end{fminipage}\vspace*{0.3cm}}

% Un alltt pas encadre
\newenvironment{nalltt}
  {\vspace*{0.3cm}\begin{nminipage}\begin{alltt}}
  {\end{alltt}\end{nminipage}\vspace*{0.3cm}}

% Une fonction encadree
\newenvironment{ffonction}[1]
  {\begin{fonction}[#1]
        \begin{fminipage}
\begin{alltt}
\rule{\linewidth}{0.5pt}}
{\end{alltt}\end{fminipage}\end{fonction}}

\newenvironment{codeonepage}
  {\begin{nminipage}\vspace*{0.2cm}\hrule\vspace*{0.1cm}
\begin{alltt}}
  {\end{alltt} \vspace*{-0.2cm}\hrule \vspace*{0.2cm} \end{nminipage}}

\newenvironment{code}
  {\vspace*{0.1cm}\hrule\vspace*{-0.1cm}\begin{alltt}}
  {\end{alltt}\vspace*{-0.2cm}\hrule \vspace*{0.1cm}}

\newcommand{\cod}[1]{{\small\textsf{#1}\xspace}}
\newcommand{\cb}[1]{\cod{#1}}
\newcommand{\trait}[1]{\cod{#1}}
\newcommand{\cls}[1]{\cod{#1}}
\newcommand{\class}[1]{\cod{#1}}

\newcommand{\ie}{i.e.,~}

%To put a return char in a code section inside a \vwspecific, \squeakspecific, ...
\chardef\caret=`\^

\newcommand{\di}{$\gg$\xspace}
\newcommand{\cat}[1]{\texttt{#1}\xspace}

\newboolean{usesqueak}
\setboolean{usesqueak}{true}
\newcommand{\smalltalk}[2]
	{\ifthenelse{\boolean{usesqueak}}
	        {#1\xspace}
	        {#2\xspace}}
	        

%%% for chapter
\font\fonttitre=cmbx12 scaled 4200
\font\fonttitrep=cmbx12 scaled 2200
%\font\fontpartie=cmbx10 scaled 1800 % Titres des parties dans la tab des mats
%\font\fontpartiegrosse=cmbx25 scaled 2000 % Titres des parties dans le texte
%\font\fontchapitre=cmbxsl10 scaled 2500 % Titres des chapitres pas utilise
\font\fontchap=cmbx12 scaled 2200% Titres des chapitres
\font\fontchapnumero=cmbx12 scaled 4000 % Titres des chapitres
	        
%%%%%%%%%%%%%%%%%%%%%%%%%%%%%%%%%%%%%%%%%%%%%%%%% Affichage d'une tete d'un chapitre sans numero (et biblio, index)
\def\@makeschapterhead#1{%
  \vspace*{2\p@}%
  {\parindent \z@ \raggedright
    \reset@font
	\vskip 18pt
	\baselineskip=16mm
	\fontencoding{OT1}\selectfont\fontchap #1\par
	\baselineskip=4mm
    \nobreak
    \vskip 30\p@
  }}
% Macro for Chapter
\makeatletter
\def\@makechapterhead#1{%
  \vspace*{-1cm}
  \vspace*{2\p@}%
  {\parindent \z@ \raggedright \reset@font
    \ifnum \c@secnumdepth >\m@ne
       \if@mainmatter
      	\flushright{\hfill\fontencoding{T1}\selectfont\fontchap\fontchapnumero\thechapter}
	%\flushright{\LARGE\@chapapp{}\thechapter}
         \par
         \vskip 9\p@
       \fi
	 \hrule height 2pt\par
	\vskip 16pt
	\baselineskip=12mm
	\fontencoding{T1}\selectfont\fontchap #1\par
	\baselineskip=4mm
    \nobreak
    \vskip 45\p@
  }}
\makeatother


\newboolean{showcomments}
\setboolean{showcomments}{false}
\ifthenelse{\boolean{showcomments}}
  {\newcommand{\mynote}[2]{
    \fbox{\bfseries\sffamily\scriptsize#1}
    {\small$\blacktriangleright$\textsf{\emph{#2}}$\blacktriangleleft$}
    % \marginpar{\fbox{\bfseries\sffamily#1}}
   }
   \newcommand{\cvsversion}{\emph{\scriptsize$-$Id: Common.tex,v 1.2 2005/11/06 13:11:22 ducasse Exp $-$}}
  }
  {\newcommand{\mynote}[2]{}
   \newcommand{\cvsversion}{}
  }
\newcommand{\here}{\mynote{***}{CONTINUE HERE}}
\newcommand\nb[1]{\mynote{NB}{#1}}
\newcommand\fix[1]{\mynote{FIX}{#1}}
% \newcommand\todo[1]{\mynote{TO DO}{#1}}
\newcommand\ab[1]{\mynote{Alex}{#1}}
\newcommand\on[1]{\mynote{Oscar}{#1}}
\newcommand\ga[1]{\mynote{Gabi}{#1}}
\newcommand\sd[1]{\mynote{Stef}{#1}}
\newcommand\lr[1]{\mynote{Lukas}{#1}}

\newboolean{toseecomment}
\setboolean{toseecomment}{false}
%%change to false to hide comment 
\newcommand{\comment}[1]{\ifthenelse{\boolean{toseecomment}}{$\blacktriangleright$ \textit{#1}$\blacktriangleleft$}{}}

\newcommand{\commented}[1]{}


% M A C R O S
% % % % % % % % % % % % % % % % % % % % % % % % % % % % % % % % % % % % %

% text styles
%\newcommand{\code}[1]{\texttt{#1}}
%\newcommand{\class}[1]{\ct{#1}}
\newcommand{\method}[1]{\ct{\##1}}
\newcommand{\bundle}[1]{\emph{#1}}
\newcommand{\package}[1]{\emph{#1}}

%% block counter
%\newcounter{block_nr}
%\setcounter{block_nr}{1}
%% block styles
%\newcommand{\block}[3]{
%	\vspace{7pt}\noindent
%	\textbf{#1~\arabic{block_nr}\addtocounter{block_nr}{1}#2}\quad#3}
%\newcommand{\exercise}[1]{\block{Exercise}{}{#1}}
%\newcommand{\exercisestar}[1]{\block{Exercise}{$^\bigstar$}{#1}}
%\newcommand{\question}[1]{\block{Question}{}{#1}}
%\newcommand{\questionstar}[1]{\block{Question}{$^\bigstar$}{#1}}

% references
\newcommand{\secref}[1]{Section~\ref{#1}}
\newcommand{\figref}[1]{Figure~\ref{#1}}





\begin{document}
\fi

\chapter{Streams: Les acc\'es}

\mainauthor{Bernard Pottier, Universit\'e de Brest}
\sd{add Unit tests}

\section{D\'efinition}

\ct{Stream} et les sous-classes de \ct{Stream} fournissent un m\'ecanisme d'acc\`es
s\'equentiel sur des objets, internes ou externes.

Lorsqu'une instance de \ct{Stream}  est utilis\'ee,  elle a seule le contr\^ole de l'objet acc\'ed\'e, qui ne devrait pas \^etre utilis\'e directement.  La resynchronisation de l'objet lu ou \'ecrit peut impliquer une op\'eration de mise \`a jour (\ct{flush},  ou \ct{commit} pour assurer la mise \`a jour d'une entr\'ee-sortie\ldots). La raison est que le m\'ecanisme s\'equentiel tamponne les donn\'ees en lecture ou en \'ecriture afin d'\'economiser les transferts.

\section{Cr\'eations et acc\`es}

\subsection{Cr\'eations}
On cr\'ee un stream\footnote{en fran\c{c}ais, un fl\^ot} en s'appuyant sur
des classes telles que \ct{ReadStream}, \ct{WriteStream}, \ct{ReadWriteStream},
lors que l'objet est interne. D'autres classes sont utilis\'ees pour les acc\`es s\'equentiels externes, pour les entr\'ee-sorties. Certain objets  savent rendre des \ct{Stream} en r\'eponse au message \ct{readStream} ou \ct{writeStream}; c'est le cas des fichiers
\ct{Filename}.

On obtient un stream en exp\'ediant le message \ct{on:} \`a la classe \`a instancier. On  passe un objet vide lorsque l'on veut \'ecrire, une collection s\'equen\c{c}able pleine lorsque l'on veut lire. Le stream s'occupe de faire grossir votre objet lorsque le besoin s'en fait sentir, en respectant sa classe.


\begin{itemize}
\item \ct{ReadStream on: 'A Plouescat, la joie eclate!'}
\item \ct{(WriteStream on: (Array new: 3))}
\end{itemize}

\subsection{Extraction du contenu, tests}

Le contenu du stream ne devrait pas \^etre acc\'ed\'e directement.
Il faut r\'ecup\'erer ce contenu en exp\'ediant le message
\ct{contents} au stream. L'exemple qui suit montre l'insertion
et l'extraction de l'objet:

\begin{scode}
\stBar monStream \stBar
monStream := ReadStream on: 'A Plouescat, la joie eclate!'.
monStream contents.   " 'A Plouescat, la joie eclate!'"
\end{scode}

On peut noter qu'il est possible de savoir si on est au bout d'une lecture
(fin de fichier, dernier \'el\'ement d'un objet), \`a l'aide du message \ct{atEnd}.
On peut se positionner en fin de stream \`a l'aide de \ct{setToEnd}.
La position dans le stream est contr\^olable \`a l'aide du message \ct{position}.

\begin{scode}
\stBar monStream \stBar
monStream := ReadStream on: 'A Plouescat, la joie eclate!'.
monStream atEnd. "(printIt) false"
monStream setToEnd ; position. "(printIt) 28"
monStream  position: 12; position "(printIt) 12"
\end{scode}

\subsection{Acc\`es en lecture}

\subsubsection{Lecture s\'equentielle d'un \'el\'ement}
La lecture est op\'er\'ee en exp\'ediant le message \ct{next}.
Le stream fait progresser son index interne et effectue d'autres
op\'erations si n\'ecessaire.

\begin{scode}
\stBar array stream \stBar
array := #( 2  3 6 7 9 0 ).
stream := ReadStream on: array.
stream next. stream next. " 3"
\end{scode}

\subsubsection{lecture d'une s\'equence}
On peut pr\'elever un nombre arbitraire d'\'el\'ements en utilisant le message \ct{nextAvailable: unEntier}. Bien entendu, on ne peut d\'epasser la capacit\'e de l'objet acc\'ed\'e, si celle-ci est born\'ee. 

\begin{scode}
\stBar array stream \stBar
array := #( 2  3 6 7 9 0 ).
stream := ReadStream on: array.
stream nextAvailable: 3. " (printIt) #(2 3 6)"
stream nextAvailable:12." (printIt) #(7 9 0)"
\end{scode}

\subsubsection{lecture d'une s\'equence sur condition}
Il est souvent int\'eressant de pr\'elever des \'el\'ements jusqu'\`a ce qu'une
condition soit obtenue.

\begin{itemize}
\item \ct{upToEnd}: \'epuise le contenu de l'objet.\\
\begin{scode}
\stBar array stream \stBar
array := #( 2  3 6 7 9 0 ).
stream := ReadStream on: array.
stream nextAvailable: 3. " (printIt) #(2 3 6)"
stream upToEnd." (printIt) #(7 9 0)" 
\end{scode}

\item \ct{through: limit}: lit jusque la prochaine occurence de l'objet \ct{limit}.\\
\begin{scode}
\stBar stream \stBar
stream := ReadStream on: 'C''est a Morlaix que l''on se plait'.
stream through:  $'.  
stream through:  $'. " (printIt)'est a Morlaix que l'''" 
\end{scode}
\end{itemize}

\subsection{Ecriture}

En \'ecriture, on envoie \ct{nextPut: unElement}, qui ins\`ere 
l'\'ele\'ement au bout de l'objet. On  peut provoquer l'insertion
d'une collection d'\'el\'ements avec {nextPutAll: uneCollection}

\begin{scode}
\stBar  stream \stBar
stream := WriteStream on: (Array new: 4).
stream nextPut: 3/2.
stream nextPutAll: #( 2  3 6 7 9 0 ).
stream contents	" #((3/2) 2 3 6 7 9 0)" 
\end{scode}

\section{Application \`a l'affichage des messages}

La fen\^etre \ct{Transcript} sert \`a l'affichage des messages.
On peut obtenir le texte contenu dans cette fen\^etre via le
message \ct{value}:\\
\ct{Transcript value asString} \\

On \'ecrit dans le Transcript comme dans un stream de caract\`eres:\\
\ct{Transcript nextPut: Character cr. \\ 
Transcript nextPutAll: 'A Plouarzel, on fait du zele'.\\
Transcript flush.}

Noter que le message \ct{flush} est indispensable pour provoquer
l'affichage imm\'ediat des caract\`eres ins\'er\'es.
Les Stream savent g\'erer l'insertion de caract\`eres non imprimables
plus simplement:\\
\ct{Transcript  cr; tab;  nextPutAll: 'A Plouarzel, on fait du zele'; \\
space;cr; flush}

\section{Cas des fichiers}

Un fichier est rep\'er\'e par un {\sl nom}, instance de la classe
\ct{Filename}. On peut choisir un nom de  fichier (\ct{String})
via un dialogue sp\'ecialis\'e:\\
\begin{scode}
\stBar nom \stBar
nom :=  Dialog requestFileName: 'nom du fichier?'
\end{scode}

Une fois ce nom choisi, on peut lui associer un fichier en 
convertissant la chaine en une instance de \ct{Filename}:\\
\begin{scode}
\stBar nom  fileName \stBar
nom :=  Dialog requestFileName: 'nom du fichier?'.
fileName := nom asFilename.
\end{scode}

On obtient ensuite ais\'ement un stream en lecture ou \'ecriture
en exp\'ediant les messages \ct{writeStream}:\\
\begin{scode}
\stBar stream \stBar
stream := 'Transcript.file' asFilename writeStream.
stream nextPutAll: Transcript value.
stream commit ; close.
\end{scode}

 ou \ct{readStream}, en lecture:\\
\begin{scode}
\stBar stream nomDuFichier \stBar
nomDuFichier := Dialog requestFileName: 'nom du fichier?'.
stream  := nomDuFichier asFilename readStream.
Transcript nextPutAll:    stream contents. 
Transcript flush. stream close.
\end{scode}


\section{Exercices}
\subsection{Analyse lexicale}
\subsubsection{En phrases}
{\sl On peut obtenir le texte de la  fen\^etre Transcript en exp\'ediant le message
\ct{value}. Construire une collection des phrases contenues dans ce texte:\\}


\begin{scode}
\stBar rStream phrases \stBar
rStream := ReadStream on:  Transcript value asString. "lecture sur la chaine"
phrases := WriteStream on: (Array new: 100). "ecriture sur  des tableaux"
[ rStream atEnd ] whileFalse: 
      	[ phrases nextPut: ( rStream upTo: $. )].
phrases contents
\end{scode}

\subsubsection{Renverser les mots}
{\sl On utilise le texte de la  fen\^etre Transcript.
Produire un texte o\`u les mots sont \'ecrits \`a l'envers\ldots} 
~\\
\begin{scode}
 \stBar rStream texte ligneStream mot \stBar
rStream := ReadStream on:  Transcript value asString. "lecture sur la chaine"
texte := WriteStream on: (String new: 1000). "ecriture du nouveau texte."
[ rStream atEnd ] whileFalse: 
      	[ ligneStream := ReadStream on: ( rStream upTo: Character cr).
                [ ligneStream atEnd ] whileFalse: 
                   [mot := ligneStream upTo: Character space . 
                    texte nextPutAll: mot reverse ; space  ].
                    texte cr.
            ].
texte contents
\end{scode}

\subsection{Analyse de code}

D'eutres classes ont des comportements de Stream. C'est par exemple le cas de l'analyseur lexical Smalltalk, qui peut \^etre r\'eutilis\'e \`a d'autres t\^aches:

\begin{scode}
\stBar texteScanner item items \stBar
item := ''. 
items := OrderedCollection new.
texteScanner :=Scanner new on:  'begin Carthago delenda est.  7 +9 = 15 . end' readStream.
[ item = 'end' ]  
   whileFalse: [ item :=texteScanner scanToken. items add: item].
items
\end{scode}
%\end{document}

\ifx\wholebook\relax\else\end{document}\fi
