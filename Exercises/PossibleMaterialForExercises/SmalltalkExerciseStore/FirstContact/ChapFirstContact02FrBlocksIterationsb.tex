% $Author: ducasse $
% $Date: 2005/11/06 13:19:03 $
% $Revision: 1.1 $

\ifx\wholebook\relax\else
\documentclass{report}
\usepackage{times}
\usepackage{graphicx}
\usepackage{ifthen}
\usepackage{xspace}
\usepackage{alltt}
\usepackage{ifpdf}
\usepackage{ifthen}
\usepackage{amsmath}
\usepackage{a4wide}

\usepackage{amssymb}
\usepackage{amsfonts}


\graphicspath{{../figures/}{figures/}{FirstContact/}{Seaside/}{Seaside/figures/}{SmallWiki/}{AdvancedSmalltalking/}{Environment/}{BotsIncExos/}{Models/}}
\usepackage[pdftex,colorlinks=true,pdfstartview=FitV,linkcolor=blue,citecolor=blue,urlcolor=blue]{hyperref}
%%%%%%%%%%%%%%%%%%%%%%%%%%%%%%%%%%%%%%%

\usepackage{ifpdf}
\ifpdf
    \pdfoutput=1
    \DeclareGraphicsExtensions{.pdf, .jpg, .png}
\else
    \renewcommand{\includegraphics}{} % No graphics in case of latex
    \DeclareGraphicsExtensions{.eps, .jpg}
\fi


%%%%%%%%%%%%%%%%%%%%%%%%%%%%%%%%%%%%%%%
\newcommand{\mainauthor}[1]{Main Author(s): #1}
\newcommand{\ducasse}{S. Ducasse, Universit\'e de Savoie, \textsf{stephane.ducasse@univ-savoie.fr}}
\newcommand{\wuyts}{R. Wuyts, Universit\'e Libre de Bruxelles, \textsf{roel.wuyts@ulb.ac.be}}
\newcommand{\bouraqadi}{N. Bouraqadi, Universit\'e Libre de Bruxelles, \textsf{bouraqadi@ensm-douai.fr}}
\newcommand{\stinckwich}{S. Stinckwich, Universit\'e de Caen, \textsf{Serge.Stinckwich@info.unicaen.fr}}
\newcommand{\bergel}{A. Bergel, Universitaet Bern, \textsf{bergel@iam.unibe.ch}}
\newcommand{\pottier}{B. Pottier, Universit\'e de Brest, \textsf{Bernard.Pottier@univ-brest.fr}}

\newcommand{\metadata}[5]{}
%\newcommand{\metadata}[5]{\begin{quote}\emph{Currently developed on: #1, Can be used for lectures using #2, latest version #4 available on #3, contact person: #5}\end{quote}}



%%%%%%%%%%%%%%%%%%%%%%%%%%%%%%%%%%%%%%%
\newboolean{hidden}
\setboolean{hidden}{true}
\newcommand{\hidden}[1]{\ifthenelse{\boolean{hidden}}{#1}{}}


\newboolean{seevwspecific}
\setboolean{seevwspecific}{true}
\newcommand{\vwspecific}[1]{\ifthenelse{\boolean{seevwspecific}}{#1}{}}

\newboolean{seesqspecific}
\setboolean{seesqspecific}{false}
\newcommand{\sqspecific}[1]{\ifthenelse{\boolean{seesqspecific}}{#1}{}}

\newboolean{seecategoryspecific}
\setboolean{seecategoryspecific}{false}
\newcommand{\categoryspecific}[1]{\ifthenelse{\boolean{seecategoryspecific}}{#1}{}}

\newboolean{seeparcelspecific}
\setboolean{seeparcelspecific}{false}
\newcommand{\parcelspecific}[1]{\ifthenelse{\boolean{seeparcelspecific}}{#1}{}}

\newboolean{seestorespecific}
\setboolean{seestorespecific}{true}
\newcommand{\storespecific}[1]{\ifthenelse{\boolean{seestorespecific}}{#1}{}}

\newboolean{seesqueaksourcespecific}
\setboolean{seesqueaksourcespecific}{false}
\newcommand{\squeaksourcespecific}[1]{\ifthenelse{\boolean{seesqueaksourcespecific}}{#1}{}}

\newboolean{seesqueakspecific}
\setboolean{seesqueakspecific}{false}
\newcommand{\squeakspecific}[1]{\ifthenelse{\boolean{seesqueakspecific}}{#1}{}}

\newcommand{\category}[0]
{\ifthenelse{\boolean{seestorespecific}}
        {package\xspace}
        {category\xspace}}

\newcommand{\stc}[1]{{\small {\sf #1}}\xspace}
\newcommand{\ST}{{\textsc Smalltalk}\xspace}
\newcommand{\tab}{\makebox[4em]{}}
\newcommand{\ttt}[1]{{\sf #1}}
\newcommand{\chev}{\ttt{>>}}
\newcommand{\superior}{\ttt{>}}
\newcommand{\assoc}{\texttt{->}}
\newcommand{\sepa}{...}
%% to get | in code
\newcommand{\stBar}{$\mid$}

%% to get >>
%%\newcommand{\sep}{$\gg$}
\newcommand{\sep}{\texttt{>>}}
%\newcommand{\sllash}{\verb=\\=}

%%code in text
\newcommand{\ct}[1]{\textsf{#1}}

\newcommand{\stMethod}[1]{\textsf{#1}}


\newcommand{\vw}{VisualWorks\xspace}
\newcommand{\VW}{VisualWorks\xspace}
\newcommand{\sq}{Squeak\xspace}
\newcommand{\store}{Store\xspace}
\renewcommand{\chaptername}{chapter}

\newcounter{exo}
%\newcommand{\exercise}[0]
%   {\stepcounter{exo}\par\vspace{0.2cm}\noindent \textbf{Exercise: \arabic{exo}.}\xspace}


\newcommand{\block}[3]{\vspace{7pt}\noindent
	\textbf{#1~\arabic{exo}\stepcounter{exo}}#2\quad#3}
\newcommand{\exercise}[1]{\block{Exercise}{}{#1}}
\newcommand{\exercisestar}[1]{\block{Exercise}{$^\bigstar$}{#1}}
\newcommand{\question}[1]{\block{Question}{}{#1}}
\newcommand{\questionstar}[1]{\block{Question}{$^\bigstar$}{#1}}

\newcommand{\exoitem}
    {\stepcounter{exo} \item[Exercise: \arabic{exo}.]}
    
%\newcommand{\fig}[3]{
%   \begin{figure}[#3]
%   \begin{center}
%   \includegraphics[width=0.8\linewidth]{#1}
%   \caption{#2.\label{fig:#1}}
%   \end{center}
%   \end{figure}
%}


\newenvironment{scode}
{\begin{alltt}\hrule\vspace{0.1cm}\sffamily}
{\vspace{0.1cm}\hrule\end{alltt}\normalsize}


%%%%%%%%%%%%%%%%%%%%%%%%%%%%%%%%%%%%%%%%%%%%%
%%%%%% Only here for backward compatibility

\newsavebox{\fminibox}
\newlength{\fminilength}

% Fait un truc encadre
\newenvironment{fminipage}[1][\linewidth]
  {\setlength{\fminilength}{#1-2\fboxsep-2\fboxrule}
        \begin{lrbox}{\fminibox}\begin{minipage}{\fminilength}}
  { \end{minipage}\end{lrbox}\noindent\fbox{\usebox{\fminibox}}}

% Pareil mais pas encadre (a utiliser pour ne pas couper une fonction

\newenvironment{nminipage}[1][\linewidth]
  {\setlength{\fminilength}{#1}
        \begin{lrbox}{\fminibox}\begin{minipage}{\fminilength}}
  { \end{minipage}\end{lrbox}\noindent\mbox{\usebox{\fminibox}}}

% Un alltt encadre
\newenvironment{falltt}
  {\vspace*{0.3cm}\begin{fminipage}\begin{alltt}}
  {\end{alltt}\end{fminipage}\vspace*{0.3cm}}

% Un alltt pas encadre
\newenvironment{nalltt}
  {\vspace*{0.3cm}\begin{nminipage}\begin{alltt}}
  {\end{alltt}\end{nminipage}\vspace*{0.3cm}}

% Une fonction encadree
\newenvironment{ffonction}[1]
  {\begin{fonction}[#1]
        \begin{fminipage}
\begin{alltt}
\rule{\linewidth}{0.5pt}}
{\end{alltt}\end{fminipage}\end{fonction}}

\newenvironment{codeonepage}
  {\begin{nminipage}\vspace*{0.2cm}\hrule\vspace*{0.1cm}
\begin{alltt}}
  {\end{alltt} \vspace*{-0.2cm}\hrule \vspace*{0.2cm} \end{nminipage}}

\newenvironment{code}
  {\vspace*{0.1cm}\hrule\vspace*{-0.1cm}\begin{alltt}}
  {\end{alltt}\vspace*{-0.2cm}\hrule \vspace*{0.1cm}}

\newcommand{\cod}[1]{{\small\textsf{#1}\xspace}}
\newcommand{\cb}[1]{\cod{#1}}
\newcommand{\trait}[1]{\cod{#1}}
\newcommand{\cls}[1]{\cod{#1}}
\newcommand{\class}[1]{\cod{#1}}

\newcommand{\ie}{i.e.,~}

%To put a return char in a code section inside a \vwspecific, \squeakspecific, ...
\chardef\caret=`\^

\newcommand{\di}{$\gg$\xspace}
\newcommand{\cat}[1]{\texttt{#1}\xspace}

\newboolean{usesqueak}
\setboolean{usesqueak}{true}
\newcommand{\smalltalk}[2]
	{\ifthenelse{\boolean{usesqueak}}
	        {#1\xspace}
	        {#2\xspace}}
	        

%%% for chapter
\font\fonttitre=cmbx12 scaled 4200
\font\fonttitrep=cmbx12 scaled 2200
%\font\fontpartie=cmbx10 scaled 1800 % Titres des parties dans la tab des mats
%\font\fontpartiegrosse=cmbx25 scaled 2000 % Titres des parties dans le texte
%\font\fontchapitre=cmbxsl10 scaled 2500 % Titres des chapitres pas utilise
\font\fontchap=cmbx12 scaled 2200% Titres des chapitres
\font\fontchapnumero=cmbx12 scaled 4000 % Titres des chapitres
	        
%%%%%%%%%%%%%%%%%%%%%%%%%%%%%%%%%%%%%%%%%%%%%%%%% Affichage d'une tete d'un chapitre sans numero (et biblio, index)
\def\@makeschapterhead#1{%
  \vspace*{2\p@}%
  {\parindent \z@ \raggedright
    \reset@font
	\vskip 18pt
	\baselineskip=16mm
	\fontencoding{OT1}\selectfont\fontchap #1\par
	\baselineskip=4mm
    \nobreak
    \vskip 30\p@
  }}
% Macro for Chapter
\makeatletter
\def\@makechapterhead#1{%
  \vspace*{-1cm}
  \vspace*{2\p@}%
  {\parindent \z@ \raggedright \reset@font
    \ifnum \c@secnumdepth >\m@ne
       \if@mainmatter
      	\flushright{\hfill\fontencoding{T1}\selectfont\fontchap\fontchapnumero\thechapter}
	%\flushright{\LARGE\@chapapp{}\thechapter}
         \par
         \vskip 9\p@
       \fi
	 \hrule height 2pt\par
	\vskip 16pt
	\baselineskip=12mm
	\fontencoding{T1}\selectfont\fontchap #1\par
	\baselineskip=4mm
    \nobreak
    \vskip 45\p@
  }}
\makeatother


\newboolean{showcomments}
\setboolean{showcomments}{false}
\ifthenelse{\boolean{showcomments}}
  {\newcommand{\mynote}[2]{
    \fbox{\bfseries\sffamily\scriptsize#1}
    {\small$\blacktriangleright$\textsf{\emph{#2}}$\blacktriangleleft$}
    % \marginpar{\fbox{\bfseries\sffamily#1}}
   }
   \newcommand{\cvsversion}{\emph{\scriptsize$-$Id: Common.tex,v 1.2 2005/11/06 13:11:22 ducasse Exp $-$}}
  }
  {\newcommand{\mynote}[2]{}
   \newcommand{\cvsversion}{}
  }
\newcommand{\here}{\mynote{***}{CONTINUE HERE}}
\newcommand\nb[1]{\mynote{NB}{#1}}
\newcommand\fix[1]{\mynote{FIX}{#1}}
% \newcommand\todo[1]{\mynote{TO DO}{#1}}
\newcommand\ab[1]{\mynote{Alex}{#1}}
\newcommand\on[1]{\mynote{Oscar}{#1}}
\newcommand\ga[1]{\mynote{Gabi}{#1}}
\newcommand\sd[1]{\mynote{Stef}{#1}}
\newcommand\lr[1]{\mynote{Lukas}{#1}}

\newboolean{toseecomment}
\setboolean{toseecomment}{false}
%%change to false to hide comment 
\newcommand{\comment}[1]{\ifthenelse{\boolean{toseecomment}}{$\blacktriangleright$ \textit{#1}$\blacktriangleleft$}{}}

\newcommand{\commented}[1]{}


% M A C R O S
% % % % % % % % % % % % % % % % % % % % % % % % % % % % % % % % % % % % %

% text styles
%\newcommand{\code}[1]{\texttt{#1}}
%\newcommand{\class}[1]{\ct{#1}}
\newcommand{\method}[1]{\ct{\##1}}
\newcommand{\bundle}[1]{\emph{#1}}
\newcommand{\package}[1]{\emph{#1}}

%% block counter
%\newcounter{block_nr}
%\setcounter{block_nr}{1}
%% block styles
%\newcommand{\block}[3]{
%	\vspace{7pt}\noindent
%	\textbf{#1~\arabic{block_nr}\addtocounter{block_nr}{1}#2}\quad#3}
%\newcommand{\exercise}[1]{\block{Exercise}{}{#1}}
%\newcommand{\exercisestar}[1]{\block{Exercise}{$^\bigstar$}{#1}}
%\newcommand{\question}[1]{\block{Question}{}{#1}}
%\newcommand{\questionstar}[1]{\block{Question}{$^\bigstar$}{#1}}

% references
\newcommand{\secref}[1]{Section~\ref{#1}}
\newcommand{\figref}[1]{Figure~\ref{#1}}





\begin{document}
\fi

\chapter{Les blocs et les enumerateurs}

%\newtheorem{exo}{Exercice}
%\title{{\bf DEUG A MIAS/SM/TI, 2$^{\grave eme}$ ann\'{e}e}\\
%TD-TP N$^0$ 1 : {\sc Les objets de Smalltalk-80}}
%\author{}
%\date{}
%\begin{document}
%\maketitle

\mainauthor{to be fixed: \pottier }
\metadata{VisualWorks}{?Squeak/VisualWorks?}{Universit\'e de Brest ---\pottier et al. }{?1.2?}{??}

%\documentclass[11pt]{article}
%\usepackage{epsf}
%\usepackage{french}
%\topmargin -2.5cm
%\addtolength{\oddsidemargin}{-1.5cm}
%\addtolength{\evensidemargin}{-1.5cm}
%\addtolength{\textheight}{7cm}
%\addtolength{\textwidth}{4cm}
%%\newtheorem{exo}{Exercice}
%\title{{\bf DEUG A MIAS/SM/STPI, 2$^{\grave eme}$ ann\'{e}e}\\
%TD-TP N$^0$ 2 : {\sc Les blocs et les \'enum\'erateurs}}
%\author{}
%\date{}

%\begin{document}
%\maketitle

\section{Commen\c cons}

On affiche une cha\^{\i}ne dans le \ct{Transcript} en lui envoyant le
message \ct{show: uneChaine}. On passe \`a la ligne dans le
\ct{Transcript} en lui envoyant le message \ct{cr}\\
Exemple~:
\ct{Transcript show: 'Salut'. Transcript cr.}\\
en utilisant la cascade (plusieurs messages envoy\'es au m\^eme objet)\\
\ct{Transcript show: 'Salut'; cr.}
 
\subsection{Les blocs}

L'\'evaluation d'un bloc sans parametre s'obtient en envoyant le message \ct{value} au
bloc. Les instructions du bloc sont ex\'ecut\'ees et le r\'esultat de la
derni\`ere instruction est retourn\'e.\\
~\\
Un bloc avec une variable s'\'evalue en envoyant le message
\ct{value: uneValeur}.\\
Un bloc avec deux variables s'\'evalue en envoyant le message\\
\ct{value: unePremiereValeur value: uneDeuxiemeValeur} 
(ceci jusqu'\`a 4 variables).

\begin{enumerate}
\item
Ecrire un bloc (sans variables) qui calcule la racine carr\'ee de 1, puis
de 2, puis de 3. Evaluer ce bloc.
\item
Ecrire un bloc avec une variable, qui renvoie le carr\'e de cette variable.  Evaluer ce bloc.
\item
Ecrire un bloc avec deux variables, qui renvoie la plus grande de ces deux
variables (utiliser la m\'ethode \ct{max:}). Evaluer ce bloc.
\end{enumerate}




\subsection{Les m\'ethodes d'intervalle}

Les nombres comprennent les messages suivants~:~
\ct{to: uneValeurArret do: unBloc}\\
et~:~
\ct{to: uneValeurArret by: unPas do: unBloc}\\
~\\
Exemple qui affiche les nombres de 1 \`a 10 dans le \ct{Transcript}\\
\ct{1 to: 10 do: [ :i \stBar Transcript show: i printString]}\\
~\\
Exemple qui affiche, par pas de 2, les nombres de 1 \`a 10 dans le
\ct{Transcript}\\
\ct{1 to: 10 by: 2 do: [ :i \stBar Transcript show: i printString]}

\begin{enumerate}
\item Convertir le nombre 65 (repr\'esentant un code ASCII) en Character
(\ct{asCharacter}), puis en Symbol (\ct{asSymbol}),
puis en String (\ct{asString}).
\item Afficher dans le \ct{Transcript} les caract\`eres compris entre les
codes ASCII 65 et 122 (bornes incluses).
\item
Afficher les nombres impairs de 1 \`a 33 dans le \ct{Transcript}
\end{enumerate}

\subsection{Les \'enum\'erateurs}
Toutes les sous-classes de \ct{Collection} comprennent ces
messages, appel\'es {\it \'enum\'erateurs}~:
\begin{itemize}
\item
\ct{do: unBloc} \'evalue \ct{unBloc} sur chaque \'el\'ement de la collection,
\item
\ct{collect: unBloc} comme \ct{do:} mais renvoie une collection des r\'esultats,
\item
\ct{select: unBloc} \'evalue \ct{unBloc} sur chaque \'el\'ement et renvoie ceux pour qui l'\'evaluation renvoie \ct{true},
\item
\ct{reject: unBloc} \'evalue \ct{unBloc} sur chaque \'el\'ement et renvoie ceux pour qui l'\'evaluation renvoie \ct{false},
\item
\ct{detect: unBloc} renvoie le premier \'el\'ement pour qui l'\'evaluation renvoie \ct{true},
\item
\ct{detect: unBloc ifNone: unAutreBloc} a le m\^eme comportement que
\ct{detect:} mais permet d'ex\'ecuter le deuxi\`eme bloc (\ct{unAutreBloc})
s'il n'y a pas d'\'el\'ement  pour qui l'\'evaluation de \ct{unBloc}
renvoie \ct{true}.
\item
\ct{inject: uneValeur into: unBlocBinaire} injecte le r\'esultat de l'ex\'ecution pr\'ec\'edente de \ct{unBlocBinaire} (un bloc \`a deux param\`etres)
dans la suivante.\\
\ct{uneCollection inject: valeur into: [arg1 arg2 \stBar ...]}\\
\ct{arg1} est initialis\'e avec \ct{valeur},\\
\ct{arg2} prend successivement la valeur de chaque \'el\'ement et \'evalue
le bloc avec cette valeur (comme un \ct{do:}),\\
\`a l'issue de l'\'evaluation courante, le r\'esultat de l'\'evaluation est
affect\'ee dans \ct{arg1}.
\end{itemize}

\subsubsection{L'\'enum\'erateur \ct{do:}}

\begin{enumerate}
\item
Faire la somme des \'el\'ements d'un tableau.
\item
La m\'ethode \ct{constantNames} envoy\'ee \`a la classe \ct{ColorValue}
renvoie un tableau {\bf constant} de symboles, chaque symbole ayant le nom
d'une couleur.

Afficher,  dans le \ct{Transcript}, les couleurs {\bf constantes} de la
classe \ct{ColorValue}
\end{enumerate}

\subsubsection{L'\'enum\'erateur \ct{collect:}}
 
\begin{enumerate}
\item
Construire un premier tableau, {\bf et avec la m\'ethode \ct{collect:}},
construire un deuxi\`eme tableau identique au premier.
\item
A partir d'un premier tableau, construire un deuxi\`eme tableau dont la valeur
de chaque \'el\'ement est le double de l'\'el\'ement correspondant dans le
premier tableau.
\end{enumerate}

\subsubsection{Autres \'enumerateurs}
\begin{enumerate}
\item
A partir du tableau  constant de symboles  ayant le nom
d'une couleur \ct{\#(noir bleu rouge rose blanc vert)}, d\'etecter la premi\`ere couleur
commen\c{c}ant par un \ct{r}\\
(on obtient le premier caract\`ere d'un symbole avec \ct{first}).
\item
A partir du tableau constant de symboles  ayant le nom
d'une couleur, construire un tableau des noms de couleur
commen\c{c}ant par un \ct{r}
\item
A partir du tableau constant de symboles  ayant le nom
d'une couleur, construire un tableau des noms de couleur
ne commen\c{c}ant pas par un \ct{r}\\
Afficher (avec un \ct{do:}) ce tableau dans le \ct{Transcript}
\end{enumerate}

\subsection{Les structures alternatives}

Smalltalk d\'efinit sur la classe \ct{True} et sur la classe \ct{False}
quatre m\'ethodes d'instance\\
\ct{ifTrue:}, \ct{ifFalse:}, \ct{ifTrue:ifFalse:}, \ct{ifFalse:ifTrue:}
\footnote{Un truc : Si on tape $<Control>t$ (ou $<Control>f$) dans une
 fen\^etre de code,\\le syst\`eme ins\`ere \ct{ifTrue:} (\ct{ifFalse:})}.

\begin{figure}[htp]
\small
\begin{center}
\begin{tabular}{|c|c|c|}
\hline
& Classe \ct{True} & Classe \ct{False}\\
\hline
M\'ethode: & \ct{ifTrue: unBloc} &  \ct{ifTrue: unBloc}\\
Action: & renvoyer l'\'evaluation de \ct{unBloc} & renvoyer \ct{nil}\\
\hline
M\'ethode: & \ct{ifFalse: unBloc} & \ct{ifFalse: unBloc}\\
Action: & renvoyer \ct{nil} & renvoyer l'\'evaluation de \ct{unBloc}\\
\hline
M\'ethode: & \ct{ifTrue: unBloc ifFalse: unAutreBloc} & \ct{ifTrue: unBloc ifFalse: unAutreBloc}\\
Action: & renvoyer l'\'evaluation de \ct{unBloc} & renvoyer l'\'evaluation de 
\ct{unAutreBloc}\\
\hline
M\'ethode: & \ct{ifFalse: unBloc ifTrue: unAutreBloc} & \ct{ifFalse: unBloc ifTrue: unAutreBloc}\\
Action: & renvoyer l'\'evaluation de \ct{unAutreBloc} & renvoyer l'\'evaluation de \ct{unBloc}\\
\hline
\end{tabular}
\end{center}
\end{figure}
Par ailleurs, les op\'erateurs logiques \ct{\&} (conjonction, le ET),
\ct{Eqv} (\'equivalence), \ct{not} (n\'egation), \ct{xor} (ou exclusif),
\ct{\stBar} (disjonction, le OU) sont d\'efinies sur les classes \ct{True}
et \ct{False}.

\begin{enumerate}
\item
Tester si un nombre est impair (en lui envoyant le message \ct{odd}) et
sonner la cloche (\sqspecific{\ct{Smalltalk beep}} \vwspecific{\ct{Screen default ringBell}}) si c'est vrai.
\item
Sans utiliser \ct{max:}, \'ecrire un bloc avec deux param\`etres qui renvoie
le maximum des deux param\`etres
\end{enumerate}

\subsection{Enum\'erateurs et alternatives}
\begin{enumerate}
\item
Faire la somme des \'el\'ements positifs d'un tableau
\item
Cr\'eer un tableau de \ct{0} ou \ct{1} \`a partir d'un tableau existant, un nombre  du tableau existant est remplac\'e par un 0 s'il est sup\'erieur
ou \'egal \`a 10, par un \ct{1} s'il est inf\'erieur \`a 10.
\end{enumerate}

\subsection{It\'eration de blocs}

Utilisation des messages \ct{ timesRepeat:}, \ct{repeat},
\ct{ whileTrue:}

\begin{enumerate}
\item Ecrire 10 fois la cha\^{\i}ne \ct{'coucou'} dans le \ct{Transcript}
(en passant \`a la ligne apr\`es chaque \ct{'coucou'}).

\item Ecrire un bloc avec une variable \ct{n} qui \'ecrit \ct{n} fois la cha\^{\i}ne \ct{'coucou'} dans le \ct{Transcript}
(en passant \`a la ligne apr\`es chaque \ct{'coucou'}). Evaluer ce bloc.

\item It\'erer avec un \ct{repeat:} un bloc qui incr\'emente un compteur de 1.
On s'arr\^ete quand le compteur est sup\'erieur \`a 10.
\item
Cr\'eer un objet de la classe \ct{Time} \`a 3 secondes du temps courant.\\
\ct{Time now addTime: (Time fromSeconds: 3)}\\
Boucler jusqu'\`a ce que le temps courant d\'epasse cet objet,
en passant \`a la ligne dans le \ct{Transcript}
\end{enumerate}

\section{Exercices}

\subsection{Les blocs}
\begin{enumerate}
\item
Ecrire un bloc avec une variable, qui renvoie la conversion en degr\'es
Celsius de cette variable (suppos\'ee \^etre en Fahrenheit).
Evaluer ce bloc (C = (5/9) (F - 32)).
\item
Ecrire un bloc avec deux variables, qui convertit ces variables en
\ct{String}, les concat\`ene en les s\'eparant avec un blanc
et renvoie le r\'esultat de la concat\'enation.
\end{enumerate}


\subsection{Les \'enum\'erateurs}

\subsubsection{L'\'enum\'erateur \ct{do:}}

\begin{enumerate}
\item
Compter le nombre d'\'el\'ements d'un tableau. V\'erifier avec \ct{size}
\item
Faire la moyenne des \'el\'ements d'un tableau.
\end{enumerate}

\subsubsection{L'\'enum\'erateur \ct{collect:}}

\begin{enumerate}
\item
Une chaine instance de la classe c\ct{String} est un tableau, donc on peut avoir son premier \'el\'ement (\ct{first}), son i-\`eme \'el\'ement (\ct{at: i}), affecter
\ct{uneValeur} dans son i-\`eme \'el\'ement \ct{at: i put: uneValeur}, etc

A partir du tableau {\bf constant} de symboles  ayant le nom
d'une couleur, construire un tableau de String o\`u le nom de la couleur
commence par une majuscule (\ct{asUppercase}).

\item
Construire un tableau dont la valeur des \'el\'ements est la somme de
l'\'el\'ement pr\'ec\'edent avec l'\'el\'ement courant.
\end{enumerate}

\subsection{Les structures alternatives}
\begin{enumerate}
\item
Sans utiliser \ct{max:}, \'ecrire un bloc avec trois param\`etres qui renvoie le maximum des trois param\`etres
\item
Ecrire un bloc avec un param\`etre, qui teste si le param\`etre est
une minuscule (\ct{isLowercase}), une majuscule (\ct{isUppercase}),
un chiffre (\ct{isDigit}) ou autre.\\
Le bloc renvoie 'minuscule' ou 'majuscule' ou 'chiffre' ou 'autre'.
\end{enumerate}

\subsection{Les intervalles}
 
\subsubsection{M\'ethodes d'intervalles}
\begin{enumerate}
\item Afficher les \ct{10} premiers carr\'es.
\item Afficher les multiples de \ct{10}.
\end{enumerate}

\subsubsection{Cr\'eer et utiliser des intervalles}

Envoy\'e \`a un nombre,  la m\'ethode \ct{to: borneSuperieure} renvoie
un \ct{Interval} allant du nombre \`a la \ct{borneSuperieure}
par pas de 1.

Envoy\'e \`a un nombre,  la m\'ethode \ct{to: borneSuperieure by: lePas}
renvoie un \ct{Interval} allant du nombre \`a la \ct{borneSuperieure}
par pas de \ct{lePas}.

Les \ct{Interval} \'etant des collections, on peut utiliser les
\'enum\'erateurs \ct{do:}, \ct{collect:}, etc.

~\\
Exemple~: \ct{tab := (1 to: 100)} cr\'ee un interval des 100 premiers entiers.

\begin{enumerate}
\item
Calculer la somme des 100 premier nombres entiers \`a l'aide d'un interval.
V\'erifier avec la formule \ct{n*(n+1)/2}

\item
Cr\'eer un interval avec les nombres de $0$ \`a $360$ de $30$ en $30$.
Utiliser cet interval pour afficher la table des sinus de $30$ degr\'es en $30$ degr\'es.
\end{enumerate}

\subsection{Enum\'erateurs et alternatives}
\begin{enumerate}
\item A partir d'un premier tableau, construire un tableau en
inversant tous les \'el\'ements qui sont des fractions.
\end{enumerate}

\subsection{Autres \'enumerateurs}
\begin{enumerate}
\item A partir d'un premier tableau, construire un deuxi\`eme tableau en
supprimant les \'el\'ements n\'egatifs.
\item D\'efinir la somme des \'el\'ements d'un tableau
avec \ct{inject:into:}.
\item D\'efinir la somme des \'el\'ements positifs d'un tableau
avec \ct{select:} puis \ct{inject:into:}.
\item D\'efinir la somme des \'el\'ements n\'egatifs d'un tableau
avec \ct{reject:} puis \ct{inject:into:}.
\end{enumerate}

\subsection{It\'eration de blocs}

\begin{enumerate}
\item
Afficher 10 \ct{*}  dans le \ct{Transcript}, puis passe \`a la ligne.

\item
Afficher 10 lignes de 10 \ct{*}  dans le \ct{Transcript} (en passant \`a
la ligne apr\`es chaque ligne de 10 \ct{*}.

\item
Ecrire un bloc \`a deux param\`etres ct{l} et \ct{c}, qui affiche
\ct{l} lignes de \ct{c} colonnes de \ct{*}
dans le \ct{Transcript} (en passant \`a la ligne apr\`es chaque
ligne).
\end{enumerate}
\ifx\wholebook\relax\else\end{document}\fi
