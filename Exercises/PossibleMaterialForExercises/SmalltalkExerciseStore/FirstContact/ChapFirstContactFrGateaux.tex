% $Author: ducasse $
% $Date: 2005/05/16 13:38:08 $
% $Revision: 1.1.1.1 $
\ifx\wholebook\relax\else
\documentclass{report}
\usepackage{times}
\usepackage{graphicx}
\usepackage{ifthen}
\usepackage{xspace}
\usepackage{alltt}
\usepackage{ifpdf}
\usepackage{ifthen}
\usepackage{amsmath}
\usepackage{a4wide}

\usepackage{amssymb}
\usepackage{amsfonts}


\graphicspath{{../figures/}{figures/}{FirstContact/}{Seaside/}{Seaside/figures/}{SmallWiki/}{AdvancedSmalltalking/}{Environment/}{BotsIncExos/}{Models/}}
\usepackage[pdftex,colorlinks=true,pdfstartview=FitV,linkcolor=blue,citecolor=blue,urlcolor=blue]{hyperref}
%%%%%%%%%%%%%%%%%%%%%%%%%%%%%%%%%%%%%%%

\usepackage{ifpdf}
\ifpdf
    \pdfoutput=1
    \DeclareGraphicsExtensions{.pdf, .jpg, .png}
\else
    \renewcommand{\includegraphics}{} % No graphics in case of latex
    \DeclareGraphicsExtensions{.eps, .jpg}
\fi


%%%%%%%%%%%%%%%%%%%%%%%%%%%%%%%%%%%%%%%
\newcommand{\mainauthor}[1]{Main Author(s): #1}
\newcommand{\ducasse}{S. Ducasse, Universit\'e de Savoie, \textsf{stephane.ducasse@univ-savoie.fr}}
\newcommand{\wuyts}{R. Wuyts, Universit\'e Libre de Bruxelles, \textsf{roel.wuyts@ulb.ac.be}}
\newcommand{\bouraqadi}{N. Bouraqadi, Universit\'e Libre de Bruxelles, \textsf{bouraqadi@ensm-douai.fr}}
\newcommand{\stinckwich}{S. Stinckwich, Universit\'e de Caen, \textsf{Serge.Stinckwich@info.unicaen.fr}}
\newcommand{\bergel}{A. Bergel, Universitaet Bern, \textsf{bergel@iam.unibe.ch}}
\newcommand{\pottier}{B. Pottier, Universit\'e de Brest, \textsf{Bernard.Pottier@univ-brest.fr}}

\newcommand{\metadata}[5]{}
%\newcommand{\metadata}[5]{\begin{quote}\emph{Currently developed on: #1, Can be used for lectures using #2, latest version #4 available on #3, contact person: #5}\end{quote}}



%%%%%%%%%%%%%%%%%%%%%%%%%%%%%%%%%%%%%%%
\newboolean{hidden}
\setboolean{hidden}{true}
\newcommand{\hidden}[1]{\ifthenelse{\boolean{hidden}}{#1}{}}


\newboolean{seevwspecific}
\setboolean{seevwspecific}{true}
\newcommand{\vwspecific}[1]{\ifthenelse{\boolean{seevwspecific}}{#1}{}}

\newboolean{seesqspecific}
\setboolean{seesqspecific}{false}
\newcommand{\sqspecific}[1]{\ifthenelse{\boolean{seesqspecific}}{#1}{}}

\newboolean{seecategoryspecific}
\setboolean{seecategoryspecific}{false}
\newcommand{\categoryspecific}[1]{\ifthenelse{\boolean{seecategoryspecific}}{#1}{}}

\newboolean{seeparcelspecific}
\setboolean{seeparcelspecific}{false}
\newcommand{\parcelspecific}[1]{\ifthenelse{\boolean{seeparcelspecific}}{#1}{}}

\newboolean{seestorespecific}
\setboolean{seestorespecific}{true}
\newcommand{\storespecific}[1]{\ifthenelse{\boolean{seestorespecific}}{#1}{}}

\newboolean{seesqueaksourcespecific}
\setboolean{seesqueaksourcespecific}{false}
\newcommand{\squeaksourcespecific}[1]{\ifthenelse{\boolean{seesqueaksourcespecific}}{#1}{}}

\newboolean{seesqueakspecific}
\setboolean{seesqueakspecific}{false}
\newcommand{\squeakspecific}[1]{\ifthenelse{\boolean{seesqueakspecific}}{#1}{}}

\newcommand{\category}[0]
{\ifthenelse{\boolean{seestorespecific}}
        {package\xspace}
        {category\xspace}}

\newcommand{\stc}[1]{{\small {\sf #1}}\xspace}
\newcommand{\ST}{{\textsc Smalltalk}\xspace}
\newcommand{\tab}{\makebox[4em]{}}
\newcommand{\ttt}[1]{{\sf #1}}
\newcommand{\chev}{\ttt{>>}}
\newcommand{\superior}{\ttt{>}}
\newcommand{\assoc}{\texttt{->}}
\newcommand{\sepa}{...}
%% to get | in code
\newcommand{\stBar}{$\mid$}

%% to get >>
%%\newcommand{\sep}{$\gg$}
\newcommand{\sep}{\texttt{>>}}
%\newcommand{\sllash}{\verb=\\=}

%%code in text
\newcommand{\ct}[1]{\textsf{#1}}

\newcommand{\stMethod}[1]{\textsf{#1}}


\newcommand{\vw}{VisualWorks\xspace}
\newcommand{\VW}{VisualWorks\xspace}
\newcommand{\sq}{Squeak\xspace}
\newcommand{\store}{Store\xspace}
\renewcommand{\chaptername}{chapter}

\newcounter{exo}
%\newcommand{\exercise}[0]
%   {\stepcounter{exo}\par\vspace{0.2cm}\noindent \textbf{Exercise: \arabic{exo}.}\xspace}


\newcommand{\block}[3]{\vspace{7pt}\noindent
	\textbf{#1~\arabic{exo}\stepcounter{exo}}#2\quad#3}
\newcommand{\exercise}[1]{\block{Exercise}{}{#1}}
\newcommand{\exercisestar}[1]{\block{Exercise}{$^\bigstar$}{#1}}
\newcommand{\question}[1]{\block{Question}{}{#1}}
\newcommand{\questionstar}[1]{\block{Question}{$^\bigstar$}{#1}}

\newcommand{\exoitem}
    {\stepcounter{exo} \item[Exercise: \arabic{exo}.]}
    
%\newcommand{\fig}[3]{
%   \begin{figure}[#3]
%   \begin{center}
%   \includegraphics[width=0.8\linewidth]{#1}
%   \caption{#2.\label{fig:#1}}
%   \end{center}
%   \end{figure}
%}


\newenvironment{scode}
{\begin{alltt}\hrule\vspace{0.1cm}\sffamily}
{\vspace{0.1cm}\hrule\end{alltt}\normalsize}


%%%%%%%%%%%%%%%%%%%%%%%%%%%%%%%%%%%%%%%%%%%%%
%%%%%% Only here for backward compatibility

\newsavebox{\fminibox}
\newlength{\fminilength}

% Fait un truc encadre
\newenvironment{fminipage}[1][\linewidth]
  {\setlength{\fminilength}{#1-2\fboxsep-2\fboxrule}
        \begin{lrbox}{\fminibox}\begin{minipage}{\fminilength}}
  { \end{minipage}\end{lrbox}\noindent\fbox{\usebox{\fminibox}}}

% Pareil mais pas encadre (a utiliser pour ne pas couper une fonction

\newenvironment{nminipage}[1][\linewidth]
  {\setlength{\fminilength}{#1}
        \begin{lrbox}{\fminibox}\begin{minipage}{\fminilength}}
  { \end{minipage}\end{lrbox}\noindent\mbox{\usebox{\fminibox}}}

% Un alltt encadre
\newenvironment{falltt}
  {\vspace*{0.3cm}\begin{fminipage}\begin{alltt}}
  {\end{alltt}\end{fminipage}\vspace*{0.3cm}}

% Un alltt pas encadre
\newenvironment{nalltt}
  {\vspace*{0.3cm}\begin{nminipage}\begin{alltt}}
  {\end{alltt}\end{nminipage}\vspace*{0.3cm}}

% Une fonction encadree
\newenvironment{ffonction}[1]
  {\begin{fonction}[#1]
        \begin{fminipage}
\begin{alltt}
\rule{\linewidth}{0.5pt}}
{\end{alltt}\end{fminipage}\end{fonction}}

\newenvironment{codeonepage}
  {\begin{nminipage}\vspace*{0.2cm}\hrule\vspace*{0.1cm}
\begin{alltt}}
  {\end{alltt} \vspace*{-0.2cm}\hrule \vspace*{0.2cm} \end{nminipage}}

\newenvironment{code}
  {\vspace*{0.1cm}\hrule\vspace*{-0.1cm}\begin{alltt}}
  {\end{alltt}\vspace*{-0.2cm}\hrule \vspace*{0.1cm}}

\newcommand{\cod}[1]{{\small\textsf{#1}\xspace}}
\newcommand{\cb}[1]{\cod{#1}}
\newcommand{\trait}[1]{\cod{#1}}
\newcommand{\cls}[1]{\cod{#1}}
\newcommand{\class}[1]{\cod{#1}}

\newcommand{\ie}{i.e.,~}

%To put a return char in a code section inside a \vwspecific, \squeakspecific, ...
\chardef\caret=`\^

\newcommand{\di}{$\gg$\xspace}
\newcommand{\cat}[1]{\texttt{#1}\xspace}

\newboolean{usesqueak}
\setboolean{usesqueak}{true}
\newcommand{\smalltalk}[2]
	{\ifthenelse{\boolean{usesqueak}}
	        {#1\xspace}
	        {#2\xspace}}
	        

%%% for chapter
\font\fonttitre=cmbx12 scaled 4200
\font\fonttitrep=cmbx12 scaled 2200
%\font\fontpartie=cmbx10 scaled 1800 % Titres des parties dans la tab des mats
%\font\fontpartiegrosse=cmbx25 scaled 2000 % Titres des parties dans le texte
%\font\fontchapitre=cmbxsl10 scaled 2500 % Titres des chapitres pas utilise
\font\fontchap=cmbx12 scaled 2200% Titres des chapitres
\font\fontchapnumero=cmbx12 scaled 4000 % Titres des chapitres
	        
%%%%%%%%%%%%%%%%%%%%%%%%%%%%%%%%%%%%%%%%%%%%%%%%% Affichage d'une tete d'un chapitre sans numero (et biblio, index)
\def\@makeschapterhead#1{%
  \vspace*{2\p@}%
  {\parindent \z@ \raggedright
    \reset@font
	\vskip 18pt
	\baselineskip=16mm
	\fontencoding{OT1}\selectfont\fontchap #1\par
	\baselineskip=4mm
    \nobreak
    \vskip 30\p@
  }}
% Macro for Chapter
\makeatletter
\def\@makechapterhead#1{%
  \vspace*{-1cm}
  \vspace*{2\p@}%
  {\parindent \z@ \raggedright \reset@font
    \ifnum \c@secnumdepth >\m@ne
       \if@mainmatter
      	\flushright{\hfill\fontencoding{T1}\selectfont\fontchap\fontchapnumero\thechapter}
	%\flushright{\LARGE\@chapapp{}\thechapter}
         \par
         \vskip 9\p@
       \fi
	 \hrule height 2pt\par
	\vskip 16pt
	\baselineskip=12mm
	\fontencoding{T1}\selectfont\fontchap #1\par
	\baselineskip=4mm
    \nobreak
    \vskip 45\p@
  }}
\makeatother


\newboolean{showcomments}
\setboolean{showcomments}{false}
\ifthenelse{\boolean{showcomments}}
  {\newcommand{\mynote}[2]{
    \fbox{\bfseries\sffamily\scriptsize#1}
    {\small$\blacktriangleright$\textsf{\emph{#2}}$\blacktriangleleft$}
    % \marginpar{\fbox{\bfseries\sffamily#1}}
   }
   \newcommand{\cvsversion}{\emph{\scriptsize$-$Id: Common.tex,v 1.2 2005/11/06 13:11:22 ducasse Exp $-$}}
  }
  {\newcommand{\mynote}[2]{}
   \newcommand{\cvsversion}{}
  }
\newcommand{\here}{\mynote{***}{CONTINUE HERE}}
\newcommand\nb[1]{\mynote{NB}{#1}}
\newcommand\fix[1]{\mynote{FIX}{#1}}
% \newcommand\todo[1]{\mynote{TO DO}{#1}}
\newcommand\ab[1]{\mynote{Alex}{#1}}
\newcommand\on[1]{\mynote{Oscar}{#1}}
\newcommand\ga[1]{\mynote{Gabi}{#1}}
\newcommand\sd[1]{\mynote{Stef}{#1}}
\newcommand\lr[1]{\mynote{Lukas}{#1}}

\newboolean{toseecomment}
\setboolean{toseecomment}{false}
%%change to false to hide comment 
\newcommand{\comment}[1]{\ifthenelse{\boolean{toseecomment}}{$\blacktriangleright$ \textit{#1}$\blacktriangleleft$}{}}

\newcommand{\commented}[1]{}


% M A C R O S
% % % % % % % % % % % % % % % % % % % % % % % % % % % % % % % % % % % % %

% text styles
%\newcommand{\code}[1]{\texttt{#1}}
%\newcommand{\class}[1]{\ct{#1}}
\newcommand{\method}[1]{\ct{\##1}}
\newcommand{\bundle}[1]{\emph{#1}}
\newcommand{\package}[1]{\emph{#1}}

%% block counter
%\newcounter{block_nr}
%\setcounter{block_nr}{1}
%% block styles
%\newcommand{\block}[3]{
%	\vspace{7pt}\noindent
%	\textbf{#1~\arabic{block_nr}\addtocounter{block_nr}{1}#2}\quad#3}
%\newcommand{\exercise}[1]{\block{Exercise}{}{#1}}
%\newcommand{\exercisestar}[1]{\block{Exercise}{$^\bigstar$}{#1}}
%\newcommand{\question}[1]{\block{Question}{}{#1}}
%\newcommand{\questionstar}[1]{\block{Question}{$^\bigstar$}{#1}}

% references
\newcommand{\secref}[1]{Section~\ref{#1}}
\newcommand{\figref}[1]{Figure~\ref{#1}}





\begin{document}
\fi

\chapter{TD1}

\mainauthor{Brest}

%% LyX 1.1 created this file.  For more info, see http://www.lyx.org/.
%% Do not edit unless you really know what you are doing.
%\documentclass[english]{article}
%\usepackage[T1]{fontenc}
%\usepackage[latin1]{inputenc}
%\usepackage{geometry}
%\geometry{verbose,letterpaper,lmargin=20mm,rmargin=20mm}
%\usepackage{babel}

%\makeatletter

%%%%%%%%%%%%%%%%%%%%%%%%%%%%%% LyX specific LaTeX commands.
%\providecommand{\LyX}{L\kern-.1667em\lower.25em\hbox{Y}\kern-.125emX\@}

%%%%%%%%%%%%%%%%%%%%%%%%%%%%%% Textclass specific LaTeX commands.
% \newenvironment{lyxcode}
%   {\begin{list}{}{
%     \setlength{\rightmargin}{\leftmargin}
%     \raggedright
%     \setlength{\itemsep}{0pt}
%     \setlength{\parsep}{0pt}
%     \normalfont\ttfamily}%
%    \item[]}
%   {\end{list}}

%\makeatother
%\begin{document}
%{\centering Licence 2 IL/IS/MI/EEA/TR 2004/2005\par}

%{\centering {\huge Langages de programmation }\huge \par}

%{\centering {\huge Langage \`a Objets - TD1 }\huge \par}


\noindent
{\bf But du TD:} Structuration d'expression Smalltalk: objet, message. Les litt\'eraux du langage. Messages unaires, binaires et \`a mots cl\`es. Les r\'egles de priorit\'es.\\
\section{Rappels}
\begin{enumerate}
\item Repondez aux questions suivantes~:
\begin{itemize}
\item Lister les diff\'erents litt\'eraux
\item Lister trois objets qui ne sont {\bf pas} des litt\'eraux
\item Lister les cat\'egories de messages. Pour chacun d'entre elles donner le nombre de param\`etres, et les signes distinctifs.
\item Lister les r\`egles de priorit\'e entre messages.

\end{itemize}
\item Si on vous donne le code suivant, trouvez les variables, les litt\'eraux, les messages, les receveurs et les param\`etres des messages, donner l'ordre d'\'evaluation des messages. 

\begin{scode}
\stBar toto titi z \stBar
toto := 34 glop glop. titi := 23 +- toto.
toto pas: #te glop:#ti pas:'tte''te' glop: $r r.
z := toto = titi
schmilblick.
((titi z z: z z: #(z)) z: 12) z: 'titi'
\end{scode}

%$
\end{enumerate} 
 On rapp\`ele que l'ensemble des messages binaires est connu par le syst\`eme. Un message binaire est compos\'e de un ou deux caract\`eres parmi les suivants:
\verb|!%&*+,-/<=>?@\~|




\section{L'objet: c'est du gateau!}

Dans cet exercice on souhaite d\'ecrire une op\'eration courante (une
recette de cuisine) en utilisant le formalisme objet message receveur
de SmallTalk. Soit la recette suivante (crumble simplifi\'e sans garniture)
: {}``\emph{Dans un plat disposez un poids \'egal de farine, beurre
et sucre (150 g), m\'elangez, puis faite cuire \`a thermostat 6 pendant
40 minutes. Le g\^ateau est pr\^et}''.

On donne les indications suivantes :

\begin{itemize}
\item Classe \ct{PlatAGateau} :

\begin{itemize}
\item message pour cr\'eer une instance de \ct{PlatAGateau} : \ct{new}
\item message pour ajouter un ingr\'edient dans une instance de \ct{PlatAGateau}
: \ct{ajouter:} avec en argument l'ingr\'edient \`a ajouter
\item message pour m\'elanger le contenu d'une instance de \ct{PlatAGateau}
: \ct{mixer}
\item message pour cuire une instance de \ct{PlatAGateau} : \ct{cuireThermostat: duree:}
avec en premier argument la valeur du thermostat et en deuxi\`eme argument
la dur\'ee en minutes
\item message pour d\'emouler le contenu d'une instance de \ct{PlatAGateau}
: \ct{d\'emouler} qui renvoi en retour une instance de la classe
\ct{Gateau}.
\end{itemize}
\item Classe \ct{Beurre} :

\begin{itemize}
\item message pour cr\'eer une instance de \ct{Beurre} : \ct{new:}
avec en argument le poids en grammes
\end{itemize}
\item Classes \ct{Farine} et \ct{Sucre} : 

\begin{itemize}
\item M�me message de cr\'eation que la pour la classe \ct{Beurre}
\end{itemize}
\end{itemize}
\begin{enumerate}
\item En utilisant les Classes et les m\'ethodes fournies, \'ecrivez en Smalltalk
la recette. Vous ferez en sorte de ne pas mettre plus d'un message
par ligne. Vous pouvez utiliser des variables locales avec des affectations.



\item Ecrivez l'\'equivalent de ce code sur une seule ligne de mani\`ere \`a avoir
directement un gateau. Prenez soin de correctement positionner des parenth\`eses si n\'ecessaire.



\end{enumerate}
\section{Priorit\'e de messages et variable temporaire}
\subsection{Fonctions trigonom\'etriques}

Un nombre (en radians) comprend les messages correspondant aux  fonctions trigonom\'etriques \ct{sin cos tan arcSin arcCos arcTan}.

On convertit un nombre de Degr\'e \`a Radian en lui envoyant le message \ct{degreesToRadians}.

\paragraph{}
Calculer (\`a l'aide d'une fonction trigonom\'etrique) le c\^ot\'e d'un carr\'e dont la diagonale mesure 1.41421 m\`etres.

\subsection{Maximum}

La m\'ethode \ct{max: unAutreNombre} appliqu\'e \`a un nombre
renvoie le plus grand des deux nombres.\\
Exemple~: \ct{2 max: 6} renvoie \ct{6}.

\begin{enumerate}
\item Calculer le maximum de la somme des couples form\'es \`a partir de trois variables  \ct{a b c} contenant des valeurs quelconques, en stockant dans des variables les r\'esultats interm\'ediaires.

\item Calculer le maximum de la somme des couples form\'es \`a partir de trois variables \ct{a b c} contenant des valeurs quelconques \textbf{sans utiliser de variables interm\'ediaires}
\end{enumerate}


\subsection{Conversion}
Le message binaire \ct{//} correspond \`a l'op\'eration de division enti\`ere. Le message binaire \ct{\\} correspond \`a  l'op\'eration de modulo (reste de la division enti\`ere).\\
On utilise ces messages pour  convertir en binaire  un nombre hexad\'ecimal (de 0 \`a 15 en base 10, de 0000 \`a 1111 en base 2).

\paragraph{}
En effectuant une s\'erie de divisions enti\`eres, les chiffres binaires
sont obtenus \textbf{de la droite vers la gauche} gr\^ace au reste de la division
enti\`ere (le modulo).

Appliquer cet algorithme pour convertir le nombre hexadecimal $(F)_{16}$ (valeur d\'ecimale 15)  en base 2.




\section{Tableau et point}
Les messages \`a mots cl\'es \ct{at: unIndex} et  \ct{at: unIndex put:uneValeur} permettent l'acc\`es en lecture ou \'ecriture d'un \'el\'ement se trouvant dans un tableau. Les messages \ct{x},  \ct{y},  \ct{x: uneValeur} et  \ct{y: uneValeur} permettent l'acc\`es en lecture ou \'ecriture de l'abscisse ou de l'ordonn\'ee d'un point.

\subsection{Analyse de code}
Rep\'erer dans les lignes de code suivantes, les messages unaires, binaires et \`a mots cl\'es. En d\'eduire l'\'evaluation des expressions et des portions de code.
 

\begin{scode}
\stBar aPoint \stBar
aPoint:= Point x:2 y:1.
aPoint x: aPoint x * 2.
aPoint
\end{scode}

\begin{scode}
\stBar x tab \stBar
tab := #(4 5.0 'toto' 1111).
x := tab at:1.
tab at: 1 put: (Fraction  numerator: x*2 denominator: 7 + x negated).
tab
\end{scode}


\subsection{Exercices sur les tableaux et les points}


\begin{enumerate}

\item Multiplier  par 2 le 2 \`eme \'el\'ement d'un tableau,
\item Remplacer la valeur du 2 \`eme \'el\'ement d'un tableau par son oppos\'e.
\item  Remplacer la valeur du 3 \`eme  \'el\'ement par la valeur du 2 \`eme
\'el\'ement.
\item  Remplacer la valeur du 3 \`eme  \'el\'ement par la somme des 2 \`eme
et 3\`eme (ancienne valeur) \'el\'ements.
\end{enumerate}

\subsection{Pour aller plus loin}

\begin{enumerate}
\item Le 2 \`eme du tableau \'etant une fraction, remplacer cette fraction par la fraction inverse dans le tableau.
\item On d\'esire d\'efinir le barycentre de points. La notion de barycentre s'applique \`a un ensemble de points effect\'es par une masse. Soit $p_{1}$,$p_{2}$,$\ldots$, $p_{n}$ un ensemble de points affect\'es des masses $m_{1}$,$m_{2}$,$\ldots$, $m_{n}$, on appelle barycentre le point $p$ d\'efini de la mani\`ere suivante:
$$m.p = \sum_{i=1}^{n} m_{i}.p_{i}\; avec\; m=\sum_{i=1}^{n} m_{i}$$
D\'efinir des expressions smalltalk permettant de d\'efinir le barycentre $p$, on prendra $n=4$ et les valeurs des masses et des points seront donn\'es dans des tableaux que l'on d\'efinira.
\end{enumerate}



\ifx\wholebook\relax\else\end{document}\fi


