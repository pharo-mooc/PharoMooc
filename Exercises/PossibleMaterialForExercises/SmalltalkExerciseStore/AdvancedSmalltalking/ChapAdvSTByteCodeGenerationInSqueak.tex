% $Author: ducasse $
% $Date: 2005/05/16 13:37:18 $
% $Revision: 1.1.1.1 $

\ifx\wholebook\relax\else
\documentclass{report}
\usepackage{times}
\usepackage{graphicx}
\usepackage{ifthen}
\usepackage{xspace}
\usepackage{alltt}
\usepackage{ifpdf}
\usepackage{ifthen}
\usepackage{amsmath}
\usepackage{a4wide}

\usepackage{amssymb}
\usepackage{amsfonts}


\graphicspath{{../figures/}{figures/}{FirstContact/}{Seaside/}{Seaside/figures/}{SmallWiki/}{AdvancedSmalltalking/}{Environment/}{BotsIncExos/}{Models/}}
\usepackage[pdftex,colorlinks=true,pdfstartview=FitV,linkcolor=blue,citecolor=blue,urlcolor=blue]{hyperref}
%%%%%%%%%%%%%%%%%%%%%%%%%%%%%%%%%%%%%%%

\usepackage{ifpdf}
\ifpdf
    \pdfoutput=1
    \DeclareGraphicsExtensions{.pdf, .jpg, .png}
\else
    \renewcommand{\includegraphics}{} % No graphics in case of latex
    \DeclareGraphicsExtensions{.eps, .jpg}
\fi


%%%%%%%%%%%%%%%%%%%%%%%%%%%%%%%%%%%%%%%
\newcommand{\mainauthor}[1]{Main Author(s): #1}
\newcommand{\ducasse}{S. Ducasse, Universit\'e de Savoie, \textsf{stephane.ducasse@univ-savoie.fr}}
\newcommand{\wuyts}{R. Wuyts, Universit\'e Libre de Bruxelles, \textsf{roel.wuyts@ulb.ac.be}}
\newcommand{\bouraqadi}{N. Bouraqadi, Universit\'e Libre de Bruxelles, \textsf{bouraqadi@ensm-douai.fr}}
\newcommand{\stinckwich}{S. Stinckwich, Universit\'e de Caen, \textsf{Serge.Stinckwich@info.unicaen.fr}}
\newcommand{\bergel}{A. Bergel, Universitaet Bern, \textsf{bergel@iam.unibe.ch}}
\newcommand{\pottier}{B. Pottier, Universit\'e de Brest, \textsf{Bernard.Pottier@univ-brest.fr}}

\newcommand{\metadata}[5]{}
%\newcommand{\metadata}[5]{\begin{quote}\emph{Currently developed on: #1, Can be used for lectures using #2, latest version #4 available on #3, contact person: #5}\end{quote}}



%%%%%%%%%%%%%%%%%%%%%%%%%%%%%%%%%%%%%%%
\newboolean{hidden}
\setboolean{hidden}{true}
\newcommand{\hidden}[1]{\ifthenelse{\boolean{hidden}}{#1}{}}


\newboolean{seevwspecific}
\setboolean{seevwspecific}{true}
\newcommand{\vwspecific}[1]{\ifthenelse{\boolean{seevwspecific}}{#1}{}}

\newboolean{seesqspecific}
\setboolean{seesqspecific}{false}
\newcommand{\sqspecific}[1]{\ifthenelse{\boolean{seesqspecific}}{#1}{}}

\newboolean{seecategoryspecific}
\setboolean{seecategoryspecific}{false}
\newcommand{\categoryspecific}[1]{\ifthenelse{\boolean{seecategoryspecific}}{#1}{}}

\newboolean{seeparcelspecific}
\setboolean{seeparcelspecific}{false}
\newcommand{\parcelspecific}[1]{\ifthenelse{\boolean{seeparcelspecific}}{#1}{}}

\newboolean{seestorespecific}
\setboolean{seestorespecific}{true}
\newcommand{\storespecific}[1]{\ifthenelse{\boolean{seestorespecific}}{#1}{}}

\newboolean{seesqueaksourcespecific}
\setboolean{seesqueaksourcespecific}{false}
\newcommand{\squeaksourcespecific}[1]{\ifthenelse{\boolean{seesqueaksourcespecific}}{#1}{}}

\newboolean{seesqueakspecific}
\setboolean{seesqueakspecific}{false}
\newcommand{\squeakspecific}[1]{\ifthenelse{\boolean{seesqueakspecific}}{#1}{}}

\newcommand{\category}[0]
{\ifthenelse{\boolean{seestorespecific}}
        {package\xspace}
        {category\xspace}}

\newcommand{\stc}[1]{{\small {\sf #1}}\xspace}
\newcommand{\ST}{{\textsc Smalltalk}\xspace}
\newcommand{\tab}{\makebox[4em]{}}
\newcommand{\ttt}[1]{{\sf #1}}
\newcommand{\chev}{\ttt{>>}}
\newcommand{\superior}{\ttt{>}}
\newcommand{\assoc}{\texttt{->}}
\newcommand{\sepa}{...}
%% to get | in code
\newcommand{\stBar}{$\mid$}

%% to get >>
%%\newcommand{\sep}{$\gg$}
\newcommand{\sep}{\texttt{>>}}
%\newcommand{\sllash}{\verb=\\=}

%%code in text
\newcommand{\ct}[1]{\textsf{#1}}

\newcommand{\stMethod}[1]{\textsf{#1}}


\newcommand{\vw}{VisualWorks\xspace}
\newcommand{\VW}{VisualWorks\xspace}
\newcommand{\sq}{Squeak\xspace}
\newcommand{\store}{Store\xspace}
\renewcommand{\chaptername}{chapter}

\newcounter{exo}
%\newcommand{\exercise}[0]
%   {\stepcounter{exo}\par\vspace{0.2cm}\noindent \textbf{Exercise: \arabic{exo}.}\xspace}


\newcommand{\block}[3]{\vspace{7pt}\noindent
	\textbf{#1~\arabic{exo}\stepcounter{exo}}#2\quad#3}
\newcommand{\exercise}[1]{\block{Exercise}{}{#1}}
\newcommand{\exercisestar}[1]{\block{Exercise}{$^\bigstar$}{#1}}
\newcommand{\question}[1]{\block{Question}{}{#1}}
\newcommand{\questionstar}[1]{\block{Question}{$^\bigstar$}{#1}}

\newcommand{\exoitem}
    {\stepcounter{exo} \item[Exercise: \arabic{exo}.]}
    
%\newcommand{\fig}[3]{
%   \begin{figure}[#3]
%   \begin{center}
%   \includegraphics[width=0.8\linewidth]{#1}
%   \caption{#2.\label{fig:#1}}
%   \end{center}
%   \end{figure}
%}


\newenvironment{scode}
{\begin{alltt}\hrule\vspace{0.1cm}\sffamily}
{\vspace{0.1cm}\hrule\end{alltt}\normalsize}


%%%%%%%%%%%%%%%%%%%%%%%%%%%%%%%%%%%%%%%%%%%%%
%%%%%% Only here for backward compatibility

\newsavebox{\fminibox}
\newlength{\fminilength}

% Fait un truc encadre
\newenvironment{fminipage}[1][\linewidth]
  {\setlength{\fminilength}{#1-2\fboxsep-2\fboxrule}
        \begin{lrbox}{\fminibox}\begin{minipage}{\fminilength}}
  { \end{minipage}\end{lrbox}\noindent\fbox{\usebox{\fminibox}}}

% Pareil mais pas encadre (a utiliser pour ne pas couper une fonction

\newenvironment{nminipage}[1][\linewidth]
  {\setlength{\fminilength}{#1}
        \begin{lrbox}{\fminibox}\begin{minipage}{\fminilength}}
  { \end{minipage}\end{lrbox}\noindent\mbox{\usebox{\fminibox}}}

% Un alltt encadre
\newenvironment{falltt}
  {\vspace*{0.3cm}\begin{fminipage}\begin{alltt}}
  {\end{alltt}\end{fminipage}\vspace*{0.3cm}}

% Un alltt pas encadre
\newenvironment{nalltt}
  {\vspace*{0.3cm}\begin{nminipage}\begin{alltt}}
  {\end{alltt}\end{nminipage}\vspace*{0.3cm}}

% Une fonction encadree
\newenvironment{ffonction}[1]
  {\begin{fonction}[#1]
        \begin{fminipage}
\begin{alltt}
\rule{\linewidth}{0.5pt}}
{\end{alltt}\end{fminipage}\end{fonction}}

\newenvironment{codeonepage}
  {\begin{nminipage}\vspace*{0.2cm}\hrule\vspace*{0.1cm}
\begin{alltt}}
  {\end{alltt} \vspace*{-0.2cm}\hrule \vspace*{0.2cm} \end{nminipage}}

\newenvironment{code}
  {\vspace*{0.1cm}\hrule\vspace*{-0.1cm}\begin{alltt}}
  {\end{alltt}\vspace*{-0.2cm}\hrule \vspace*{0.1cm}}

\newcommand{\cod}[1]{{\small\textsf{#1}\xspace}}
\newcommand{\cb}[1]{\cod{#1}}
\newcommand{\trait}[1]{\cod{#1}}
\newcommand{\cls}[1]{\cod{#1}}
\newcommand{\class}[1]{\cod{#1}}

\newcommand{\ie}{i.e.,~}

%To put a return char in a code section inside a \vwspecific, \squeakspecific, ...
\chardef\caret=`\^

\newcommand{\di}{$\gg$\xspace}
\newcommand{\cat}[1]{\texttt{#1}\xspace}

\newboolean{usesqueak}
\setboolean{usesqueak}{true}
\newcommand{\smalltalk}[2]
	{\ifthenelse{\boolean{usesqueak}}
	        {#1\xspace}
	        {#2\xspace}}
	        

%%% for chapter
\font\fonttitre=cmbx12 scaled 4200
\font\fonttitrep=cmbx12 scaled 2200
%\font\fontpartie=cmbx10 scaled 1800 % Titres des parties dans la tab des mats
%\font\fontpartiegrosse=cmbx25 scaled 2000 % Titres des parties dans le texte
%\font\fontchapitre=cmbxsl10 scaled 2500 % Titres des chapitres pas utilise
\font\fontchap=cmbx12 scaled 2200% Titres des chapitres
\font\fontchapnumero=cmbx12 scaled 4000 % Titres des chapitres
	        
%%%%%%%%%%%%%%%%%%%%%%%%%%%%%%%%%%%%%%%%%%%%%%%%% Affichage d'une tete d'un chapitre sans numero (et biblio, index)
\def\@makeschapterhead#1{%
  \vspace*{2\p@}%
  {\parindent \z@ \raggedright
    \reset@font
	\vskip 18pt
	\baselineskip=16mm
	\fontencoding{OT1}\selectfont\fontchap #1\par
	\baselineskip=4mm
    \nobreak
    \vskip 30\p@
  }}
% Macro for Chapter
\makeatletter
\def\@makechapterhead#1{%
  \vspace*{-1cm}
  \vspace*{2\p@}%
  {\parindent \z@ \raggedright \reset@font
    \ifnum \c@secnumdepth >\m@ne
       \if@mainmatter
      	\flushright{\hfill\fontencoding{T1}\selectfont\fontchap\fontchapnumero\thechapter}
	%\flushright{\LARGE\@chapapp{}\thechapter}
         \par
         \vskip 9\p@
       \fi
	 \hrule height 2pt\par
	\vskip 16pt
	\baselineskip=12mm
	\fontencoding{T1}\selectfont\fontchap #1\par
	\baselineskip=4mm
    \nobreak
    \vskip 45\p@
  }}
\makeatother


\newboolean{showcomments}
\setboolean{showcomments}{false}
\ifthenelse{\boolean{showcomments}}
  {\newcommand{\mynote}[2]{
    \fbox{\bfseries\sffamily\scriptsize#1}
    {\small$\blacktriangleright$\textsf{\emph{#2}}$\blacktriangleleft$}
    % \marginpar{\fbox{\bfseries\sffamily#1}}
   }
   \newcommand{\cvsversion}{\emph{\scriptsize$-$Id: Common.tex,v 1.2 2005/11/06 13:11:22 ducasse Exp $-$}}
  }
  {\newcommand{\mynote}[2]{}
   \newcommand{\cvsversion}{}
  }
\newcommand{\here}{\mynote{***}{CONTINUE HERE}}
\newcommand\nb[1]{\mynote{NB}{#1}}
\newcommand\fix[1]{\mynote{FIX}{#1}}
% \newcommand\todo[1]{\mynote{TO DO}{#1}}
\newcommand\ab[1]{\mynote{Alex}{#1}}
\newcommand\on[1]{\mynote{Oscar}{#1}}
\newcommand\ga[1]{\mynote{Gabi}{#1}}
\newcommand\sd[1]{\mynote{Stef}{#1}}
\newcommand\lr[1]{\mynote{Lukas}{#1}}

\newboolean{toseecomment}
\setboolean{toseecomment}{false}
%%change to false to hide comment 
\newcommand{\comment}[1]{\ifthenelse{\boolean{toseecomment}}{$\blacktriangleright$ \textit{#1}$\blacktriangleleft$}{}}

\newcommand{\commented}[1]{}


% M A C R O S
% % % % % % % % % % % % % % % % % % % % % % % % % % % % % % % % % % % % %

% text styles
%\newcommand{\code}[1]{\texttt{#1}}
%\newcommand{\class}[1]{\ct{#1}}
\newcommand{\method}[1]{\ct{\##1}}
\newcommand{\bundle}[1]{\emph{#1}}
\newcommand{\package}[1]{\emph{#1}}

%% block counter
%\newcounter{block_nr}
%\setcounter{block_nr}{1}
%% block styles
%\newcommand{\block}[3]{
%	\vspace{7pt}\noindent
%	\textbf{#1~\arabic{block_nr}\addtocounter{block_nr}{1}#2}\quad#3}
%\newcommand{\exercise}[1]{\block{Exercise}{}{#1}}
%\newcommand{\exercisestar}[1]{\block{Exercise}{$^\bigstar$}{#1}}
%\newcommand{\question}[1]{\block{Question}{}{#1}}
%\newcommand{\questionstar}[1]{\block{Question}{$^\bigstar$}{#1}}

% references
\newcommand{\secref}[1]{Section~\ref{#1}}
\newcommand{\figref}[1]{Figure~\ref{#1}}





\begin{document}
\fi



\chapter{Generating Bytecodes}

\mainauthor{Marcus Denker}
\sd{to dry we need more information...}

\section*{Introduction}

Generating bytecode can be done with {\em IRBuilder}. The following exercise uses the IRBuilder
of Squeak 3.7.

Here is the the example from the lecture again: 

\begin{scode}
irMethod:= IRBuilder new
  rargs: #(self);      "receiver and args"
  pushTemp: #self;	
  send: #truncated;
  returnTop;
  ir.
  
aCompiledMethod  := irMethod compiledMethod.
\end{scode}

The variable {\em aCompiledMethod} now contains the generated compiled method. This method can
be executed:

\begin{scode}
aCompiledMethod valueWithReceiver:3.5 arguments: #()
\end{scode}

\section{Expressions}
With the help of IRBuilder, generate a method that calculates the
expression {\tt (3 + 4) factorial} and returns the result.

% Solution:
%irMethod:= IRBuilder new
%  rargs: #(self);      "receiver and args"
%  pushLiteral: 3;	
%  pushLiteral: 4;	
%  send: #+;
%  send: #factorial;
%  returnTop;
%  ir.


\section{Parameters}
Change the code that you wrote for the first exercise to use parameters
instead of hard coded numbers. So in the end you will have a method
that requires two arguments. If executed with 

\begin{scode}
aCompiledMethod valueWithReceiver: nil arguments: #(3 4)
\end{scode}

the result should be 7.

%irMethod:= IRBuilder new
%  rargs: #(self a b);      "receiver and args"
%  pushTemp: #a;	
%  pushTemp: #b;	
%  send: #+;
%  send: #factorial;
%  returnTop;
%  ir.

\section{Loops}
The Squeak bytecode has support for jumps. Jumps are used to implement conditionals and
loops in an efficient way.

Generate a compiledMethod with {\tt  IRBuilder} that outputs the numbers 1 to 10 on the Transcript window. 

%
%irMethod:= IRBuilder new
%  rargs: #(self i);      "receiver and args"
%  pushLiteral: 0;	
%  storeTemp: #i;
%  popTop;
%  jumpBackTarget: #back;
%  pushTemp: #i;
%  pushLiteral: 10;	
%  send: #<;
%  jumpAheadTo: #continue if: false;
%  pushLiteral: Transcript;
%  pushTemp: #i;
%  pushLiteral: 1;
%  send: #+;
%  storeTemp: #i;
%  send: #asString;
%  send: #show:;
%  popTop;
%  jumpBackTo: #back;
%  jumpAheadTarget: #continue;
%  returnTop;
%  ir.



\section{Instance Variables}
Generate a method that adds two instance variables and returns the result. Test the code by running
it on a Point, e.g., 3@4.

%irMethod:= IRBuilder new
%  rargs: #(self);      "receiver and args"
%  pushReceiver;
%  getField: 1;	
%  pushReceiver;
%  getField: 2;	
%  send: #+;
%  returnTop;
%  ir.



\section{Installing a Method in a Class}
Find a way to add the method from Exercise 4 to the class Point with the name {\tt returnSum}. After
that, the following test should be green:

\begin{scode}
testReturnAdd
   self assert: (1@2) returnSum = 3.
   self assert: (3@4) returnSum = 7.
\end{scode}

%Behavior>>addSelector: selector withMethod: compiledMethod 

\chapter{Bytecode Analysis}

%%%%%%%%%%%%%%%%%%%

\section{Counting Number of Executed Bytecodes}
Look at the method \ct{tallyInstructions:} in the class \ct{ContextPart} (class-side):
\begin{scode}
	"This method uses the simulator to count the number of occurrences of
	each of the Smalltalk instructions executed during evaluation of aBlock.
	Results appear in order of the byteCode set."
\ttt{|} tallies \ttt{|}
	tallies := Bag new.
	thisContext sender
		runSimulated: aBlock
		contextAtEachStep:
			[:current | tallies add: current nextByte].
	^tallies sortedElements

	"ContextPart tallyInstructions: [3.14159 printString]"
^anArray at: 2\end{scode}

The method \ct{runSimulated: aBlock contextAtEachStep: [:current| ...]} execute \ct{aBlock} and for each bytecode executed in this block or in called methods, the second argument is evaluated with an instance of one of the subclasses of \ct{ContextPart} as the argument.

Write a similar method named \ct{numberOfBytecodeExecuted: aBlock} that returns the number of bytecode executed when evaluating the provided block. For instance:
\begin{scode}
ContextPart numberOfBytecodeExecuted: [3.14159 printString]
==> 1029
\end{scode}

In total, the expression \ct{3.14159 printString} is evaluated by executing 1029 bytecodes.

%
%numberOfBytecodeExecuted: aBlock
%        "This method uses the simulator to count the number of occurrences of
%        each of the Smalltalk instructions executed during evaluation of
%aBlock. 
%        Results appear in order of the byteCode set."
%        | num | 
%        num _ 0.
%        thisContext sender
%                runSimulated: aBlock
%                contextAtEachStep:  
%                        [:cur | num := num + 1].
%        ^num                    
%        
%        "ContextPart numberOfBytecodeExecuted: [3.14159 printString]"
% 


%%%%%%%%%%%%%%%%%%%

\section{Methods Coverage Analysis}

Getting information about methods that are currently needed to perform a computation is often difficult to obtain with languages like Java. However this information can easily retrieved in Smalltalk.

\subsection*{Number of Methods Invoked}
Create a method \ct{numberOfInvokedMethods: aBlock} that return the number of all the methods invoked when \ct{aBlock} is evaluated.
\begin{scode}
ContextPart numberOfInvokedMethods: [3.14159 printString]
==> 38
\end{scode}

%numberOfMethodCovered: aBlock    
%	| m | 
%	m := Set new.
%     thisContext sender
%		runSimulated: aBlock
%		contextAtEachStep:  
%			[:cur | m add: cur sender].
%	^m size
%        
%	"ContextPart numberOfMethodCovered: [3.14159 printString]"
% 


\subsection*{Set of Methods Covered}
We are now interested in the methods name.
\begin{scode}
ContextPart methodCovered: [3.14159 printString]
==> #('ContextPart class>>DoIt' 'Object>>printString'
 'Object>>printStringLimitedTo:' ... )
\end{scode}



%methodCovered: aBlock    
%	| m | 
%	m := OrderedCollection with: ''.
%     thisContext sender
%		runSimulated: aBlock
%		contextAtEachStep:  
%			[:cur | |s| s := (cur mclass name , '>>', cur selector). (m last ~= s) ifTrue: [m add: s]].
%	m removeFirst.
%	^ m asArray
%	
%	"ContextPart methodCovered: [3.14159 printString]"
 

\subsection*{Bytecode Covered}
Let's focus on bytecode. When a method is invoked, not all the bytecode contained in this method are executed. For instance, when executing  \ct{3.14159 printString} the method \ct{on:} defined in the class \ct{WriteStream} is executed, but only 90\% of its bytecode are executed.
\begin{scode}
ContextPart bytecodeCovered: [3.14159 printString]
==>  #(#('WriteStream>>on:' 90) #('LimitedWriteStream>>nextPut:' 69)
#('Object>>species' 100) ...)
\end{scode}
%
%bytecodeCovered: aBlock    
%	| m m2 cm v n | 
%	m := Dictionary new.
%     thisContext sender
%		runSimulated: aBlock
%		contextAtEachStep:  
%			[:cur | 
%				|s| 
%				s := cur mclass >> cur selector. 
%				(m includesKey: s) ifFalse: [m at: s put: Set new].
%				
%				(m at: s) add: cur pc].
%	
%	m2 := OrderedCollection new.
%	m associationsDo: [:assoc| |s|
%		cm := assoc key.
%		s := cm endPC - cm initialPC + 1.
%		s ~~ 0 ifTrue: [
%			v := (assoc value size * 100 / s) asInteger.
%			v > 100 ifTrue: [self halt].
%			n := assoc key who.
%			m2 add: { n first name, '>>', n second. v}]].
%	^ m2 asArray
%	"ContextPart bytecodeCovered: [3.14159 printString]"
% 



\ifx\wholebook\relax\else\end{document}\fi
